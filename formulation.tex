\section{Modeling of Duplicate Bug Reports Detection}
%\section{Problem Formulation}
\label{formulation}

%Since the model is adapted from RTM~\cite{???}, in the text, we use
%some notations originated in~\cite{???}

\vspace{0.04in}
\noindent\textbf{Overview.} This section describes our formulation for
the problem of detecting duplicate bug reports. In our approach, each
system is considered to have a number of (technical) aspects. The bug
database is considered as a collection of bug reports. Each aspect of
the system is considered as a topic of that collection, represented
via a set of certain words/terms. Each bug report is considered as a
textual document containing a number of words/terms to report on some
of those technical aspects. Among them, some aspects might be
incorrectly implemented with respect to the system's requirements,
thus, causing the bugs being reported. Two bug reports are considered
to be duplicate if they similarly describe the same buggy
topic(s). From now on, we use the terms (technical) ``aspect'' and
``topic'' interchangeably. The same treatment is for ``bug report''
and ``document'', and ``word'' and ``term''.

Since a document is a bug report, their topics with higher proportions
are more likely to be buggy or highly relevant to the buggy
topics. Other topics might have zero or very low proportion. If two
documents are duplicate, i.e. reporting the same buggy topics, their
corresponding topic proportions of those common topics would be high
in both reports. If two documents are not duplicate bug reports, the
two topic proportions might not be much similar (i.e. the distribution
values of common topics might not be high in both documents).

To formulate the duplicate bug reports detection problem, we develop a
parameterized, probabilistic, generative model. We call it
\emph{incremental Relational Topic Model (iRTM)} because we extend the
original RTM~\cite{RTM} for this detection problem. The key ideas of
{\model} include:

1) it considers bug reports as the observations which arise from a
generative, probabilistic process with hidden variables. For this
problem, the collection of bug reports including duplicate ones is
modeled as to be {\em generated} by {\model} model;

2) via training on historical data, for example the already-identified
   as duplicate bug reports, it {\em estimates} those hidden variables
   using posterior inference;

3) then it situates new data into the estimated model, i.e., it will
   predict {\em how likely a new bug report and an existing one are
   generated by the estimated model} to determine whether they are
   duplicate ones; and

4) The trained model is quickly updated via our incremental algorithms
   (Section~IV) without fully re-training as new bug reports arrive.

%how likely
%the new bug report is generated by the estimated model.

\begin{figure}
\centerline{\epsfxsize=3.2in \epsffile{illustration2.eps}}
\caption{Topic Modeling~\cite{lda}}
\label{topicmodel}
\end{figure}

%\vspace{0.04in}
%\noindent {\bf Topic Modeling.} 

\subsection{Relational Topic Modeling}

For topic modeling in bug reports, we use LDA~\cite{lda} and adapt
RTM~\cite{RTM}. Figure~\ref{topicmodel} shows the details.  The global
parameters include a vocabulary containing $V$ words available to be
used in the bug reports and a set of $K$ topics. The topic indexed by
$t$ ($t=1..K$) is modeled by a vector $\phi_t$ of size $V$, in which
$\phi_{t,x}$ is the probability that word $x$ in the vocabulary is
used to describe topic $t$.

A document $d$ generated by this model is considered as a sequence of
$N_d$ words, describing a mixture of such topics. The mixture, called
{\em topic distribution}, is represented by a vector $\theta_d$ of
size $K$. $\theta_{d,t}$ is the proportion of topic $t$, i.e. there
are expectedly $\theta_{d,t}*N_d$ words in document $d$ describing
topic $t$. Each position within $d$ is used to describe one topic. The
topic assignment for such positions is represented by a vector $z_d$
of size $N_d$, in which $z_{d,n}$, having value in 1..$K$, is the
index of the topic that the word $n^{th}$ in document $d$ describes.

%-------------------------------------------------------------------
\vspace{0.04in}
\noindent {\bf Duplication Indicator.} 
RTM is adapted to model the duplication relation among bug
reports. For two bug reports $d$ and $d'$, a {\em duplication
indicator} $y_{d,d'}$, will be set to 1 if they are duplicate, and 0
otherwise. Because we determine the duplication of two bug reports
based on how similarly they describe the same buggy topics, we define
a function $\psi(d,d')$ to measure the topic-based similarity of two
documents and determine the value of $y_{d,d'}$ based on
$\psi(d,d')$. That is, the higher $\psi(d,d')$ is, the higher
probability that $d$ and $d'$ are duplicate bug reports. In this
context, $\psi$ is called the {\em duplication indicating
function}. As suggested in~\cite{RTM}, we use the following function:
$$\psi(d, d') = exp({\sum\nolimits_{k=1}^K(\eta_k.\theta_{d,k}.\theta_{d',k})+
\nu})$$ in which $\eta_k$s are the weighted parameters and $\nu$ is a
smoothening parameter. As seen, this function measures the similarity of
the two documents via their topics. First, it calculates the
similarity of their topics via a weighted product of the corresponding
topic distribution vectors $\theta_d$ and $\theta_{d'}$. Then, it uses
an exponential function to amplify such similarity of those
vectors to calculate the desired topic-based similarity ($\psi(d,d')$)
between those two documents.

If two documents report the same buggy topics, their corresponding
topic distribution values of those common topics would be high in both
vectors, leading to the high weighted product and high duplication
indicating value returned in the formula via the exponential function.
If two documents are not duplicate bug reports, i.e. the distribution
values of common topics might not be high in both vectors, thus, the
weighted product and the duplication indicating value would be low.

%The correlation between this function and the duplication indicator
%could be understood intuitively as follows. Since a document is a
%bug report, their topics with higher proportions are more likely to be
%buggy or highly relevant to the buggy topics. Other topics might have
%zero or very low distribution. If two documents are duplicate,
%i.e. reporting the same buggy topics, their topic distributions might
%not be the same. However, their corresponding topic distribution
%values of those common topics would be high in both vectors, leading
%to the high weighted product and high duplication indicating value
%returned in the formula via the exponential function. Otherwise, if
%two documents are not duplicate bug reports, the two vectors might not
%be much similar (i.e. the distribution values of common topics might
%not be high in both vectors), thus, the weighted product and the
%duplication indicating value would be low.

\vspace{0.03in}\noindent\textbf{Generation Process.}
Formally, based on its parameters, RTM considers that the bug reports
and their duplication indicators are generated according to the
following probabilistic process:

%the process to generate the bug reports and their duplication
%indicators is as the following:

1. Choose the vocabulary of size $V$, the number of topics $K$, the
   number of documents $M$, and the other per-collection parameters
   $\alpha$, $\beta$, $\eta$, and $\nu$.

2. Choose the per-topic term distributions. For each topic $t$ in
1..$K$, draw $\phi_t$ following Dirichlet distribution
$Dir(\beta,V)$, i.e., $\phi_t \sim Dir(\beta,V)$. Each sample of
$Dir(\beta,V)$ is a vector of $V$ non-negative elements that are
summed up to 1.

3. For each document $d$ in the range of 1..$M$:

3.1. Choose the per-document topic distribution $\theta_d$ of document
     $d$: $\theta_d \sim Dir(\alpha, K)$. $\theta_d$ is a vector $K$
     non-negative elements, summed up to 1, and $\theta_{d,t}$ is the
     relative proportion of the words that are used for topic $t$ in
     document $d$.

3.2. Choose the size of document $N_d$ following Poisson
     distribution as in ~\cite{RTM}.

3.3. Choose the per-document topic assignment $z_d$ of document
     $d$. $z_d$ is a vector of $N_d$ integer elements. The $n^{th}$
     element is the index of the topic assigned for the $n^{th}$ word
     in $d$. Since $d$ has topic distribution $\theta_d$,
     $z_{d,n} \sim Multinomial(\theta_d)$.

3.4. Generate the words $w_d$ of document $d$. $w_d$ is a vector of $N_d$
     elements, in which $w_{d,n}$ is the index in the vocabulary of
     the concrete word at the $n^{th}$ position of $d$. Since the
     word at the $n^{th}$ position is assigned to topic $z_{d,n}$,
     $w_{d,n}$ is drawn based on the per-topic term distribution
     $\phi_{z_{d,n}}$: $w_{d,n} \sim Multinomial(\phi_{z_{d,n}})$.

3.5. Generate the duplication indicators of document $d$ with respect
     to other generated documents. For each generated document $d'$
     with topic proportion $\theta_{d'}$, if $\psi(d, d')$ is sufficiently large
     then $y_{d,d'} = 1$, otherwise $y_{d,d'} = 0$.

%Figure ? illustrates the generating process of two documents and the duplicating indicator between them.

%This process models how the collection of bug reports including
%duplicate ones can be generated in reality.

Note that this is a hypothesized process from the point of view of a
machine learning mechanism for the generation of bug reports including
the duplicated ones. This process is used for the training and
prediction purpose in {\model}. It does not imply a real-life process
of bug reporting.

%\vspace{0.04in}
%\noindent {\bf Incremental Training and Inferring.}

\subsection{{\model} for Incremental Training and Inferring}

Let us describe our extension to RTM for the bug report duplication
detection. In {\model}, we use the above process to model the
generation of the collection of bug reports including the duplicate
ones. Therefore, we could detect duplicate bug reports by following
step 3.5. However, we can observe only the words in bug reports,
i.e. vector $w_d$, and some manually detected duplication indicators
$y_{d,d'}$ for some pairs of bug reports that have been recorded in
the history. Thus, to detect the unobserved duplication indicators of
other pairs, we need to infer the hidden parameters of the collection
including 1) the per-topic term distribution $\phi_t$ for the whole
collection, 2) the per-document topic distribution $\theta_d$, and 3)
the per-word topic assignment $z_d$ of each document. This inferring
process must also take into account new information on reports and
their duplications to update its inferred parameters. The reason is
that in software evolution, new bug reports are filed continually and
duplicate reports are also newly identified (for both old and new
reports). Thus, there are three phases in {\model}:

\begin{enumerate}

%\vspace{0.05in}
\item {\bf Initial training}.  Documents ($w_d$s) and recorded
duplication indicators ($y_{d,d'}$s) are provided. {\model} is
trained to get topic structures ($\phi_t$ for each $t$) of the
collection, and that of each document ($\theta_d$ and $z_d$ for
each~$d$).

%\vspace{0.05in}
\item {\bf Detecting}.  In this phase, a document $d$ is provided. This
document might be an already-filed or a newly filed bug report. We
detect whether it is a duplicate report of another filed report. Thus,
the model is used to infer the topic structures of the report
($\theta_d$ and $z_d$, if needed, e.g. for a new report), and more
importantly, its duplication indicators ($y_{d,d'}$) to all other
documents. The inferred indicators, i.e. potential duplications, are
reported to the users for manual verification.

%\vspace{0.04in}
\item {\bf Updating}.
%In this phase, we have newly available information that is not used in
%the last training phase. This information includes new bug reports and
%duplication indicators (including the indicators verified after the
%detecting phase, or new indicators that are manually identified by
%users). To keep the model up-to-date with that newly
%available information, we update its recently trained parameters
%such as $\phi_t$, $\theta_d$, $z_d$ (will be described next).
In practice, the bug reports are constantly filed. New information on
the duplicate reports is also provided.  This information includes new
bug reports and duplication indicators (including the indicators
verified after the detecting phase, or new indicators that are
manually identified by users). For example, the users could manually
identify some new duplicate bug reports. They could verify some of the
automatically detected duplications by our tool as true
duplications. The model, thus, needs to be updated with newly
available information. Otherwise, if the initially trained model is
used to process new data, that model might not fit well.

A naive updating method is to completely re-train the model on both
already-trained and newly available data. Since the new data is
provided with high rate and volume, and the trained data is also of
high volume, this naive approach would be very inefficient. On the
other hand, if we train the model with only new data, we might miss
the potential duplications between the new and the existing bug
reports. Thus, our balanced approach is to select a representative
portion of existing data to re-train with new data and use the trained
information to update the global parameters (e.g. per-topic word
distribution) as suggested in~\cite{canini09}. We will explain how our
model {\model} will update its recently trained parameters such as
$\phi_t$, $\theta_d$, $z_d$ in the next section.

\end{enumerate}

%To keep the model up-to-date with that newly
%available information, we update its recently trained parameters
%such as $\phi_t$, $\theta_d$, $z_d$ (will be described next).
