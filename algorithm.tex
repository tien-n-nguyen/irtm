\section{Algorithms for initial training, detecting, and incremental updating}
\label{algorithm}

This section presents our algorithms for three aforementioned tasks:
initial training, detecting, and incremental updating. Before
presenting the algorithms (Section IV.B), let us describe a core step
in all 3 algorithms, that is, to determine the hidden (latent) topic
assignment $z_d$ of each bug report $d$ based on the provided data,
i.e. the words $w_d$ of each bug report and some recorded duplication
indicators $y_{d,d'}$s. Once the topic assignments ($z_d$s) are
inferred for all documents, we could estimate their topic proportions
($\theta_d$), the per-topic word distribution $\phi_t$ for each topic
$t$, the duplication indicating function $\psi(d,d')$, and thus, the
duplication indicators $y_{d,d'}$ of all pairs of documents.

\subsection{Sampling-based Inference of Hidden Topic Assignment}

%Direct inference of the hidden topic assignment for each document is
%intractable, thus, 

Instead of using variational inference for the hidden topic assignment
as in LDA~\cite{lda} and RTM~\cite{RTM}, we choose an approximate
approach called {\em Gibbs sampling}~\cite{gibb}. That is, the
posterior probability $P(z_d|w_d, y_{d,d'})$ is estimated by randomly
choosing the value for each element $z_{d,n}$ of $z_d$,
element-by-element, based on a distribution calculated from other
sampled values, until $P(z_d|w_d, y_{d,d'})$ is stationary. The choice
of Gibbs sampling also allows us to efficiently perform incremental
training for new data.

%A detailed description of this technique can be found in~\cite{gibb}.

%That is, we estimate the posterior probability/likelihood $P(z_d|w_d,
%y_{d,d'})$ by randomly choosing the value for each
%element $z_{d,n}$ of $z_d$, element-by-element, based on a distribution
%calculated from other sampled values, until such likelihood
%($P(z_d|w_d, y_{d,d'})$) is stationary.

%Once $z_d$ of each document is being sampled, $\theta_d$ and $\phi_t$ are also estimated based on such $z_d$s.

%In this task, the input includes: term vector $w_d$ of each bug
%report, and duplication indicators $y_{d,d'}$ of some pairs of
%manually-identified duplicate bug reports, and chosen parameters
%$\alpha, \beta, \eta$. The output includes: topic assignment vector
%$z_d$ and topic proportion $\theta_d$ of each bug report, per-topic
%term distribution $\psi_t$ of each topic, and duplication indicating
%values $\psi(\theta_d,\theta_d')$ for every pair of reports.

Specifically, the sampling-based inference of $z_d$ for each document
$d$ is as follows. The elements of $z_d$ are initially assigned with
random values. Then, each of its elements $z_{d,n}$ is sampled based
on a conditional distribution calculated from the most recent sampled
values of \emph{all other} elements, denoted by $z_{d,-n}$, and other
given information, i.e. $w_d$ and all $y_{d,d'}$s. Let us use $P(n,t)
= P(z_{d,n}=t|z_{d,-n},w_d,y_{d,d'})$ to denote the probability that
topic $t$ is assigned for the $n^{th}$ position of $d$, given
all such information. This probability depends on:

\begin{enumerate}

\item The assignments in other positions: $P(z_d[n]=t|z_d[-n])$,

\item The word $x$ chosen for position $n$: $P(w_{d,n} = x|z_{d,n} = t,z_{d,-n},w_{d,-n})$,

\item The duplication indicators of $d$ to all other documents:
   $P(z_{d,n} = t|z_{d,-n},z_{d'}, y_{d,d'})$ of all documents~$d'$.

\end{enumerate}

\vspace{0.03in}
\noindent {\bf Computation.} Those probabilities are computed as
follows:

1. $z_{d,n}$, by the generative process, is drawn based on $\theta_d$,
   while $\theta_d$ is drawn from $Dir(\alpha)$ and can be estimated
   based on $z_{d,-n}$. Let us use $\aleph(z,t)$ to denote the count
   function, i.e. the number of elements of vector $z$ having the
   value $t$. Due to the properties of Dirichlet distribution, we have
$$P_1(n,t) = P(z_{d,n} = t|z_{d,-n}) \approx \frac {\aleph(z_{d,-n},t) + \alpha} {\sum\nolimits_{k = 1}^K ({\aleph(z_{d,-n},k)} + \alpha)}$$
$$ = \frac {\aleph(z_{d,-n},t) + \alpha} {N_d - 1 + K.\alpha}$$

In other words, the topic proportion $\theta_d$ of document $d$ is
estimated by counting the assignments on $z_{d,-n}$, smoothened by
$\alpha$. Then, the probability of topic assignment $z_{d,n}$ is computed based
on such proportion.

% the ?expectation? of the distribution of $\theta_d$!!!
%(Tung: I'm not much sure about this).

2. $w_{d,n}$, by the generative process, is drawn based on $\phi_t$ if
   $z_{d,n} = t$. Let us use $w_{d,-n}(t)$ to denote the vector of words
   in $w_{d,-n}$ that are assigned to topic $t$. Due to the properties
   of Dirichlet distribution, we have
$$P_2(n,t)=P(w_{d,n}=x|z_{d,n}=t,z_{d,-n},w_{d,-n})$$ $$\approx \frac
{\aleph(w_{d,-n}(t),x) + \beta} {\sum\nolimits_{y = 1}^V
{(\aleph(w_{d,-n}(t),y) + \beta)}}= \frac {\aleph(w_{d,-n}(t),x) +
\beta} {N^{-}_{d,t} + V.\beta}$$ in which $N^{-}_{d,t}$ is the number
of words in $w_{d,-n}$ that are assigned to topic $t$.
In other words, when the position $n$ is assigned to topic $t$, we
emphasize only to the words in the document at other positions
assigned to topic $t$ (i.e. $w_{d,-n}(t)$), estimate the selection
probability of each word based on $w_{d,-n}(t)$ (smoothened by
$\beta$), and calculate the likelihood that the word $x$ has been
chosen for position $n$ (observed from data) based on that
distribution.

3. $y_{d,d'}$ is drawn based on $\psi(d, d')$ by the generative
   process. Given $z_{d,-n}$ and $z_{d'}$, we can estimate $\theta_d$
   and $\theta_d'$. Thus:
\[
\tiny
P_3(n,t,d') = P(z_{d,n} = t|z_{d,-n},z_{d'}, y_{d,d'}=1) \propto \] \[ \frac {P(y_{d,d'}=1|z_{d,n}=t, z_{d,-n},z_{d'})} {P(y_{d,d'}=1|z_{d,-n},z_d')} = \frac {\psi(d,d')} {\psi(d_{-n},d')}
\]

Since $\theta_d$ and $\theta_{d'}$ could be estimated based on $z_d$ and $z_{d'}$, by definition, $\psi(d,d') \approx exp(S)$ and $\psi(d_{-n},d') \approx exp(S')$ with

$S = {\sum_{k=1}^K(\eta_k.\frac {\aleph(z_d,k)} {N_d}. \frac {\aleph(z_{d'},k)} {N_{d'}} + \nu)}$

\noindent and

$S_{-n} = {\sum_{k=1}^K(\eta_k.\frac {\aleph(z_{d,-n},k)} {N_d}. \frac {\aleph(z_{d'},k)} {N_{d'}} + \nu}))$

\noindent Therefore, 
$$\frac {\psi(d,d')} {\psi(d_{-n},d')} = exp(S - S_{-n})
= exp({\eta_t.\frac 1 {N_d}.\frac {\aleph(z_{d'},t)} {N_{d'}}})$$

\noindent because there is only one difference between $S$ and $S_{-n}$ at position $n$ and $z_{d,n} = t$).
The formula means that, if $d$ and $d'$ are recorded as duplicate
reports, the higher proportion of topic $t$ in $d'$, the higher the
probability that a position in $d$ is assigned to topic $t$.

%[TUNG: We have the last term from reducing the formula of $\psi$.
%Note that: the topic assignments in the [tu so] and [mau so] has only a difference at for topic $t$ at location $n$].

Using all the above, we can calculate the distribution for sampling
topic assignment at each position of $d$ as:

$$P(n,t) \propto P_1(n,t).P_2(n,t). \prod_{d': y_{d,d'}=1} P_3(n,t,d')(*)$$

The last product is applied for all documents initially specified as
the duplications of $d$, i.e. all $d'$s such that $y_{d,d'} = 1$. This
is used for the pairs of reports that were recorded as duplicate
ones. For the documents having no observed duplication indicators to
$d$, we consider their indicators as un-observed, thus, ignore their
impact in topic assignment of $d$.

\subsection{Algorithms for Three Phases}

Let us describe the algorithms for three core tasks in {\model}.

%the usage of
%{\model}.

%^\vspace{0.05in}\noindent\textbf{1. Training with Initial Data}
\subsubsection{Training with Initial Data}

Using the previous core step, the initial training algorithm is as
follows:

\begin{itemize}

\item Step 1. Initialize randomly the values for all $z_d$s.

\item Step 2. For each $d$, sample $z_d$ element-by-element following
distribution (*) until it is stationary.

\item Step 3. Estimate the topic proportion of each document: $\theta_{d,t}
\approx \frac {\aleph(z_d,t)} {N_d}$.

\item Step 4. Estimate the word distribution for each topic: $\phi_{t,x}
\approx \frac {\sum_{d}{\aleph(z_d(t),x)}} {\sum_{d} {N_d(t)}}$,
i.e. count the assignments $N_d(t)$ of topic $t$ in each document $d$,
sum up for the whole collection, and estimate the proportion of
assignments using word $x$.

\end{itemize}

%\vspace{0.05in}\noindent\textbf{2. Detecting Duplicate Bug Reports}

\subsubsection{Detecting Duplicate Bug Reports}

Given a new report $d$, we need to determine if it is
duplicate of a bug report(s) in the historical data. The
detection process is as follows:

\begin{itemize}

\item Step 1. Initialize the values for $z_d$.

\item Step 2. Sample $z_d$ following the distribution (*) until it is
stationary.

\item Step 3. Estimate the topic proportion of $d$: $\theta_{d,t} \approx
\frac {\aleph(z_d,t)} {N_d}$.

\item Step 4. For any other document $d'$ in the collection, calculate the
duplication indicating function $\psi$ on the pair $(d, d')$ to infer
their duplication indicator $y_{d,d'}$.

\end{itemize}

Note that, $d$ could be a bug report in the historical data. In this
case, we could detect the not-yet-identified duplicate reports among
the filed ones, and the steps 1-3 are not needed since they were
done in the training phase.

%\vspace{0.05in}\noindent\textbf{3. Updating with Newly Available Data}

\subsubsection{Updating with Newly Available Data}
\label{updating-algo}

%In practice, the bug reports are constantly filed. New information on
%the duplicate reports is also provided. For example, the users could
%manually identify some new duplicate bug reports. They could verify
%some of the automatically detected duplications by our tool as true
%duplications. The model, thus, needs to be updated with this newly
%available information. Otherwise, if the initially trained model is
%used to process new data, that model might not fit well.

%because it is trained with old data, which might be totally irrelevant
%in the new data.

%A naive updating method is to completely re-train the model on both
%already-trained and newly available data. Since the new data is
%provided with high rate and volume, and the trained data is also of
%high volume, this naive approach would be very inefficient. However,
%if we train the model with only new data, we might miss the potential
%duplications between the new and the existing bug reports. Thus, our
%balanced approach is to select a representative portion of existing
%data to re-train with new data and use the trained information to
%update the global parameters (e.g. per-topic word distribution) as
%suggested in~\cite{canini09}.

Based on this strategy, our algorithm for incremental updating is as
follows. The input of our algorithm includes the input and output from
the last training step. In addition, the input also includes new data,
i.e. newly filed bug reports and newly provided duplication
indicators (could be among either recorded or new reports). The output
is similar as in the initial training phase.

\begin{itemize}

\item Step 1. Select all the existing/historical bug reports that are
indicated as duplications via the newly provided duplication indicators --
if exists.

\item Step 2. Randomly select another portion of historical bug reports
until having a total of $r\%$ of existing bug reports. When a bug report
is selected, we also select all its duplicate reports.

\item Step 3. Combine the selected bug reports and newly filed ones, and then
train the model on this data using the algorithm in Section~3.1.

\item Step 4. Re-estimate the parameters of the model.

\end{itemize}

$\theta_d$ is only re-estimated for the reports selected in
re-training. However, $\phi_t$ needs to be re-estimated for all bug
reports. This is done without re-counting the non-selected documents
by storing the counting values from the last training. For example,
let $n_0$, $n_1$, and $n_2$ are the numbers of words $x$ assigned to
topic $t$ in the last trained data, the old data selected for
retraining, and the re-trained data, respectively. The number of words
$x$ assigned to topic $t$ after retraining is $n_0 - n_1 + n_2$. Since
$n_0$ is stored, we only need to count $n_1$ and $n_2$ on the data
used for re-training, which is much smaller than the whole data.
