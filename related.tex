\section{Related Work}

A related work to {\model} is the Support Vector Machine (SVM)-based
approach from Sun {\em et al.}~\cite{davidlo10}.  In the training
phase of their model, all the pairs of duplicate bug reports are
formed and considered as the positive samples. All other pairs of
non-duplicate bug reports are used as the negative samples. For each
sample (i.e. a pair of reports), a total of 54 features are
extracted. Each feature is represented by the sum of all inverse
values of the frequencies of documents containing terms or bi-grams
(i.e. two consecutive terms) that appear in the summaries and/or
descriptions of both bug reports in the sample. All positive and
negative pairs/samples are used to train and derive the parameters of
a SVM model. In the prediction phase, as a new report arrives, it
would be paired with all existing reports. Then, each pair would be
fed into the trained SVM model and be classified as positive or
negative. Positive pairs imply duplicate reports. Finally, if a new
bug report is predicted by SVM model as duplicate, it is compared
with existing bug reports to determine its master bug report.

There are key advances of {\model} over their SVM model. First, their
model is not suitable for software evolution. It cannot work in an
{\em incremental} manner. For new bug reports, their model requires
complete re-training. As the project evolves, the bug report data gets
increasingly large, the training set continually grows, and the
re-training time will keep increasing significantly. The reason is
that as more reports are added, the numbers of (negative and positive)
pairs/samples will dramatically increase due to the nature of pairing
in the SVM model. With the speed of 3-5 new bug reports per hour
(e.g. in Eclipse)~\cite{davidlo10}, their time cost of for fully
re-training is much higher than {\model}'s updating time.

%not quite practical. In contrast, {\model} can incrementally update
%its parameters in very short time as new data is available.

Another disadvantage of their approach is that, the SVM model predicts
that a new report is a duplication of multiple existing bug reports
but requires a second phase to rank which one is more likely than
others. That is, after predicting the duplication of a new bug report,
their tool performs a second phase to determine the list of potential
master reports. This adds extra computational time. In contrast,
{\model} is able to rank potential master reports based on the
probabilities of generating corresponding pairs of reports. The reason
is that {\model} treats the problem of detecting duplication reports
as a {\em ranking problem}, while their SVM-based approach considers
it as a {\em classification one}. Moreover, in contrast to our
generative model, their model is SVM, a discriminative model. The
quality of their results depends very much on the positive and
negative sets of samples. Because the percentage of duplicate bug
reports is much smaller than that of non-duplicate ones in the
project, the negative set will grow faster and their approach faces
the issue of un-balance between positive and negative samples. With
the generative approach, {\model} does not have to deal with negative
and positive sample sets. Instead, it decides the probability of
generating a pair of duplicate bug reports.

To overcome that, Sun {\em et al.}~\cite{sun-ase11} introduced REP, a
novel IR technique that extends BM25F to consider the long bug reports
and the meta-data such as the reported product, component, and
version. They showed that REP outperformed the state-of-the-art ML
approaches in both accuracy and efficiency. We did not compare with REP
because it is IR-based while we used machine learning.
%XW
%In this work, we combine BM25F with our novel topic model, T-Model, to
%address the cases where duplicate reports have different terms for the
%same technical issue. To our knowledge, DBTM is the first work in
%which topic-based features are used with IR to support the detection
%of duplicate bug reports.
%
Jalbert and Weimer~\cite{weimer08} use a binary classifier model for
predicting duplicate bug reports. They utilizes a linear regression
over {\em textual} features of bug reports computed from the
frequencies of terms in bug reports.
%%%To make a binary classifier, they specify an output value cutoff over
%%%such features that distinguishes between duplicate and non-duplicate
%%%status.
Similar to Sun {\em et al.}'s model, this model requires complete
re-training for new bug reports.
%, which is not quite efficient in practice.
Moreover, their model relies solely on {\em textual similarity}, while
{\model} focuses more on the underlying {\em technical topics} of bug
reports to determine the duplications.
%Finally, their model is discriminative, thus, facing the issue of
%unbalanced positive and negative sample sets of bug reports.

One of the first techniques to detect duplicate bug reports is Runeson
{\em et al.}'s~\cite{runeson07}. In contrast to aforementioned machine
learning (ML) approaches, Runeson {\em et al.}  utilize a natural
language processing (NLP) approach.
%The bug reports are parsed, stemmed, and the stopwords are
%removed.
Each report is modeled by a vector of textual features. The
feature of such a vector at a position corresponding to a term is
computed based on Term frequency-Inverse document
frequency~\cite{salton73}. Vector similarity is used to measure the
similarity among bug reports.
%Given a bug report under investigation, their tool returns similar bug
%reports based on the vector similarities between the new report and
%the existing ones.
Hiew~\cite{hiew06}'s approach for duplicate bug report detection is
based on information retrieval (IR) as in Runeson's. However, it uses
incremental clustering for further grouping of duplicate reports.
%is based on incremental clustering, which is quite similar to
%information retrieval. The main difference is that Hiew�s approach
%further considers the detected duplicate bug-report pairs/groups as
%clusters. Thus, when calculating similarities between a new report
%and existing bug reports, each detected cluster is considered as a
%whole rather than as several individual existing bug reports. That
%is to say, for each detected cluster, this approach will calculate one
%similarity between the new report and the detected cluster instead
%of calculating several similarities between the new report and all
%the reports in the cluster.
Comparing to those approaches, {\model} operates at a higher
abstraction level by comparing the underlying technical topics in
reports, instead of their terms. Moreover, ML approaches have been
shown to outperform NLP/IR approaches~\cite{davidlo10}. Wang {\em et
al.}~\cite{taoxie08} combine NLP with execution trace information in a
report.
%%%They utilize both Tf and Idf for textual feature extraction.
DBTM~\cite{ase12} uses a combination of IR and topic modeling.
We do not use IR in this work, therefore, we did not compare our work
with DBTM.
%Despite performance improvement, their approach is not always
%applicable in the cases where execution information considered in
%their tool is not available.
%%%and hard to collect,
%%%especially for binary programs. Sun {\em et al.}~\cite{davidlo10}
%%%reported that the percentage of reports having execution information
%%%is very low (0.83\%).

Other researchers also focus on bug reports. It is suggested that
duplicate bug reports complement to one another to help
in bug fixing~\cite{bettenburg-icsm-2008}. 
%%Bettenburg {\em et al.}~\cite{rahul08} analyzed information mismatch
%%between what developers need and what users supply to determine good
%%properties in bug reports. 
Structural information from bug reports has been shown to be
useful~\cite{bettenburg-msr08,ko06}.
%%Hooimeijer and Weimer~\cite{weimer-ase07} develop a statistic-based
%%model to automatically predict the quality of bug reports.
%Ko {\em et al.}~\cite{ko06} perform linguistic analysis on bug reports
%and suggest more structure for their contents.
Other researchers categorize bug reports based on types, quality, or
severity~\cite{rahul08,weimer-ase07,anvik06,andy-pod03,cubranic04,menzies08,bettenburg-eclipse07,weimer06,lucca02,fischer03,Sandusky04bugreport}.
%However, none of them addresses the automatic detection of duplicate
%bug reports.
From bug reports, prediction tools~\cite{weiss07,skim06} can tell
whether a bug could be resolved with certain fixing time. Approaches
for automatic assignments of bug fixers include
~\cite{anvik06,Canfora05howsoftware}.
%canfora-sac06}.
%%Other approaches aim to study the relationships among bug
%%reports~\cite{fischer03,Sandusky04bugreport}. 
Gethers and Poshyvanyk~\cite{gethers10} utilize RTM in capturing the
latent topics in classes and their relationships.





%and Whitehead claim that the time it takes to fix a
%bug is a useful software quality measure [8]. They measure
%the time taken to fix bugs in two software projects. We
%predict whether a bug will eventually be resolved as a duplicate
%and are not focused on particular resolution times or
%the total lifetime of real bugs.




