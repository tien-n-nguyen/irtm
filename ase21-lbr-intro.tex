\section{Introduction}
\label{intro}

%In software development and maintenance, fixing software defects
%(often called bugs) is crucial in software development to produce
%high-quality products. Bug fixing could occur during development or in
%post-release time.

Developers, testers, or users of a software system encounter the
detects and note its incorrect behaviors that do not follow the
requirements or their expectations. They could report such defects in
a bug repository.
%that often called bug-tracking database.
%
%The process of bug fixing continues with the developers analyzing the
%phenomenon, locating the defective code, correcting them, and
%committing the fixed code to the projects' repository.
%
Users could interact with a software system in many different ways,
thus, occasionally, a bug could be filed by multiple reporters. That
leads to the {\em duplicate bug reports} for the same bugs. Because
bug reports keep being filed everyday with high pace (e.g., in
Eclipse, 3-5 new bug reports are often filed every hour), it is~vital
to detect if a new bug report is a duplicate one. This
%Detecting whether a new bug report is a duplicate one is crucial
%because it helps 
helps reduce the maintenance efforts from developers (e.g., if the bug
is already fixed) and provides additional information in the bug
fixing process (e.g., if the bug is not yet fixed).
%
However, such automated detection of duplicate bug reports is not
trivial. With different input data, usage environments or scenarios,
an erroneous behavior might expose as different phenomena (e.g.,
different outputs, traces, or screen views). Moreover, different
reporters might use different terminologies and styles, or report on
different elements to describe the same phenomena. As a result,
duplicate bug reports could be textually dissimilar.

%[Tung: we should show this in the motivating examples]

%To automate the detection process of duplicate bug reports, we propose
%a probabilistic approach. In our approach, each bug report is
%considered as a textual document describing one or more technical
%aspects/functionality of a system, in which some of them might be
%erroneously implemented. The reports similarly describing the same
%erroneous technical aspects are considered as duplicate
%ones. Considering the technical aspects as the topics of those
%text-based bug reports, we utilize Relational Topic Model
%(RTM)~\cite{RTM}, a probabilistic, generative model, to formulate
%their topic structures and duplication indicators. We use Gibbs
%sampling~\cite{gibb} to train the model on historical data with
%identified duplicate bug reports and then detect other
%not-yet-identified duplicate ones.

% iRTM
%With software evolution, new reports are continually filed and new
%duplicate information could also be provided. The trained model should
%be updated with that new information. Therefore, we extend RTM into
%{\em incremental} RTM (iRTM) in which the trained model can be updated
%using a combination of a portion of existing data (historical reports
%and duplication indicators) and newly provided data (i.e., all new
%reports and additional duplication indicators). This helps save time
%on complete re-training on all old and new data, which could be very
%costly as the number of bug reports increases with a relatively high
%pace.

%

%Our empirical evaluation on several large, real-world systems shows
%that {\model} outperforms the state-of-the-art approach from Sun {\em
%et al.}~\cite{davidlo10} in terms of both accuracy and time
%efficiency. It can achieve up to 90\% top-10 accuracy, with updating
%time within 0.14 seconds per new bug report on average. The updating
%time for our model with new data is about 5-8 times smaller than the
%re-training time of the Sun {\em et al.}'s approach
%in~\cite{davidlo10}.

%
%The contributions of this paper include:

%\begin{enumerate}

%\item {\model}, an extended model from RTM to formulate the problem of
%  detecting duplicate bug reports. {\model} captures semantically the
%  technical topics in the bug reports and formulates the semantic
%  similarity measure among duplicate reports based on such topic
%  structures.

%\item Incremental algorithms for {\model} in a) training the model on
%historical bug reports and identified duplications, b) detecting
%not-yet-identified duplicate reports, and c) updating the trained
%model when new data is available. Such incremental solution supports
%well the detection of duplicate bug reports in evolving software.

%\item An empirical evaluation showing the accuracy, scalability, and
%time efficiency of {\model}.

%\end{enumerate}

%The next section presents a motivating example. Section~3 describes
%the details of our model. Section~4 presents the algorithm for
%incremental training of the model and detection of duplicate bug
%reports. Section 5 discusses our evaluation. Related work is in
%Section 6, and conclusions appear last.
