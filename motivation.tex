%\begin{figure}[t]
%\sf
%\small
%\textbf{ID}:000002; \textbf{CreationDate}:Wed Oct 10 20:34:00 CDT 2001; \textbf{Reporter}:Andre Weinand

%\textbf{Summary}: Opening repository resources doesn't honor type.

%\textbf{Description}:Opening repository resource always open the default text editor and doesn't honor any mapping between resource types and editors. As a result it is not possible to view the contents of an image (*.gif file) in a sensible way.
%\rm
%\caption{Bug report BR2 in Eclipse project}
%\label{fig:br1}
%\end{figure}


\begin{figure}[t]
\sf
\small
\textbf{ID}:488105; \textbf{Reported:}:2016-02-19 08:46 EST; \\\textbf{Reporter}:Marc Dumais

\textbf{Summary}:  [memory] Traditional and Floating Point Renderings: view is initially empty

\textbf{Description}: This happens in both the Memory and Memory Browser views, for Traditional and Floating Point renderings. 

When a new Memory Browser view is created, or when a new rendering is created in the Memory view using one of these renderings, the view is initially not populated. It then becomes populated when a refresh occurs, for example by re-sizing the view. See attached screenshot for an example.
\rm
\caption{Bug report 488105~\cite{bug488105} in Eclipse project}
\label{fig:br1}
\end{figure}


\section{Motivating Example}
\label{sec:example}

Let us explain an example of duplicate bug reports that motivate our
approach. Generally, a bug report is a record in a bug-tracking
repository of a project, containing the descriptions on the
bug(s). Typically, a bug record contains the following important
fields 1) a unique ID of the report (\textbf{\sf ID}), reported date
(\textbf{\sf ReportedDate}), the reporter (\textbf{\sf Reporter}), and
a short summary (\textbf{\sf Summary}) and a full description
(\textbf{\sf Description}).

\vspace{0.05in}\noindent\textbf{Observations on a bug report}
Figure~\ref{fig:br1} displays an example of an already-fixed bug
report in Eclipse project. As shown, this bug report was assigned the
ID of 488105 and reported on 02/19/2016 by Marc Dumais for a bug on
Eclipse. It described that the initial view or initial rendering for
memory browser is not populated. When it becomes populated when a
refresh occurs. Analyzing the textual description, we have the
following observations:

%functions/technical aspects/concerns/featues 

\begin{enumerate}

\item This report is about two technical functions in Eclipse:
  \emph{memory browser} (MEM) and \emph{view management} (VM) of
  software artifacts. In general, MEM involves the operations such as
  \emph{create}, \emph{render}, and \emph{populate}. VM involves the
  operations such as {\em refresh}, {\em resize}, {\em rendering},~etc.

\item The bug occurred in the code implementing memory browsing. That
  is, the operation {\em rendering}  on an initial view
   was incorrectly implemented.
%Technically, the system maintains no mappings between resource types
%and editors, thus, it uses the default text editor to open all kinds
%of resource.

\item In this bug report, the technical function MEM can be recognized
  in its contents via the words that are relevant to memory browsing
  such as \code{rendering}, \code{view}, \code{create}, \code{populate},
  \code{floating}, and \code{memory}.
  Similarly, the description also contains relevant terms to VM such
  as \code{refresh}, \code{resize}, and \code{rendering}. Note that,
  some words such as \code{rendering} and \code{view} are used to
  describe both functions MEM and VM. If considering bug reports as
  textual documents, we can view the described technical functions as
  the \textbf{topics} of those documents.

\end{enumerate}

%\begin{figure}
%\sf
%\small
%\textbf{ID}:009779; \textbf{CreationDate}:Wed Feb 13 15:14:00 CST 2002; \textbf{Reporter}:Jeff Brown

%\textbf{Resolution}:DUPLICATE

%\textbf{Summary}: Opening a remote revision of a file should not always use the default text editor.

%\textbf{Description}: \code{OpenRemoteFileAction} hardwires the editor
%that is used to open remote file to
%\code{org.eclipse.ui.DefaultTextEditor} instead of trying to find an
%appropriate one given the file's type.

%You get the default text editor regardless of whether there are
%registered editors for files of that type -- even if it's binary. I
%think it would make browsing the repository or resource history
%somewhat nicer if the same mechanism was used here as when files are
%opened from the navigator. We can ask the Workbench's
%\code{IEditorRegistry} for the default editor given the
%filename. Use text only as a last resort (or perhaps because of a
%user preference).  \rm
%\caption{Bug report BR9779, a duplication of bug report BR2 in Eclipse}
%\label{fig:br2}
%\end{figure}

\begin{figure}
\sf
\small
\textbf{ID}:Bug 493035; \textbf{Reported}: 2016-05-04 18:12 EDT; \textbf{Reporter}:Marc-Andre Laperle

\textbf{Resolution}:DUPLICATE

\textbf{Summary}: [GTK3] CDT Memory Browser initially blank

\textbf{Description}: 
Using I20160504-0035
CDT master as of today
Ubuntu 16.04 and Fedora 23 (GTK 3.18.9)
C/C++ Memory View Enhancements feature installed

1. Create a hello world C++ project
2. Start debugging
3. Open the memory browser
4. Enter main in the address field. The memory view is blank.

If the view is resized, it gets populated correctly. This works correctly in GTK2.
\caption{Bug report 493035~\cite{bug493035}, a duplication of bug report 488105 in Eclipse}
\label{fig:br2}
\end{figure}

\noindent\textbf{Observations on a duplicate bug report}
Figure~\ref{fig:br2} presents bug report ID 493035, filed on
05/04/2016 by a different reporter, Marc-Andre Laperle. This report
was determined by Eclipse's developers as reporting the same bug as in
bug report 488105. Analyzing the contents of 493035 and comparing to
those of 488105, we can see that

%have the following observations:

\begin{enumerate}


\item BR 493035 also describes two aspects/functions: MEM - memory
  browsing and VM - view management. MEM was also reported to contain
  bug(s).

\item The terms that are used to describe MEM are similar to those in
  BR 488105, e.g., \code{memory}, \code{browser}, and
  \code{initially}. The terms describing VM are somewhat similar
  such as \code{view}, \code{resize}, and \code{populate}.

\item BR 493035 provides additional information about the bug such as
  the OS and the feature under which the bug occurred (Ubuntu, Fedora,
  Memory View Enhancements, address field). It also provides the steps
  to reproduce the bug. The text contains those additional terms. It
  mentioned the cases that the system worked correctly such as after
  being resized, getting populated, or in GTK2.

%It notifies that \code{OpenRemoteFileAction}, the class responsible
%for opening a remote file, is directly associated with
%\code{org.eclipse.ui.DefaultTextEditor}, i.e., it always uses the
%default editor to open a remote file. The report also provides a
%fixing suggestion: asking Workbench's \code{IEditorRegistry} for the
%default editor given the filename.

%Based on those two examples and five observations, we imply/conclude
%that:
%1. Duplicate bug reports do exist in real-world software development (the bug discussed here is also duplicately reported in BR000094 and BR015392). This is because the software system tend to be used by many independent people in many different environment and usage settings. Thus, an existing bug is easily seen/experienced and reported by several people.

\end{enumerate}

\vspace{0.03in}\noindent\textbf{Implications} Several researchers have
reported the benefits of the detection of duplicate bug
reports. First, the duplicate bug reports provide various different
view points and usage experience in different scenarios. Those give
different angles of the same bug(s), helping developers in the project
to debug and fix the defect(s) more quickly. Second, knowing the new
report is a duplication and was already fixed in a previous one helps
reduce the cost of redundant fixing efforts.

%The detection of such duplicate bug reports has several benefits in
%software development and maintenance. First, the duplicate bug
%reports, reported by people with different points of view and
%experience could provide different kinds of information about the
%bug(s), thus, help in the debugging and fixing process. Importantly,
%detecting duplicate bug reports would help developers to avoid
%redundant bug fixing efforts.

However, manual detection of duplicate bug reports is highly
time-consuming. For a large-scale project, the number of bugs and the
bug reporting rate are fairly high. For example, in Eclipse, there are
currently more than 533K bug reports and there are from 2-5 newly
filed bug reports every hour. To detect whether a new bug report is a
duplication of some existing bug report, one would need to analyze and
compare it with all those bug reports, both new and existing
ones. Detection on the whole dataset would result in the analysis of
$O(N)$ and the comparison of $O(N^2)$, with $N$ is the total number of
bug reports. Therefore, an automatic detection of duplicate bug
reports is highly desirable.

The above example shows us that the detection of duplicate bug reports
could be based on their technical topics, rather than the concrete
terms/words that are used. Intuitively, topics are \emph{latent}, {\em
  semantic} features, while terms are \emph{visible, textual} features
of the documents. One could expect that the former would describe the
similarity of the documents more accurate than the latter. For
example, BR 488105 and BR 493035 describe the same topic, but they
might use \emph{different} terms for the same topic. In BR 493035, the
words \code{blank}, {\em debugging}, {\em view enhancements}, are used
to describe VM, while they do not appear in BR 488105.

%Thus, term-based assessment of those two documents might be less
%accurate/effective than topic-based assessment.

%Actually, using tfidf, a term-weighting scheme, we have calculate the
%cosine similarity of BR000002 and BR009779 to be 0.5???, too small to
%be considered as similar documents. In our approach, the topic-based
%probability that they are duplicated is estimated as 0.8???.
% topic EDIT in BR2 >> topic EDIT in BR1. (Buggy one!!!)

Based on aforementioned observations, we propose to use a topic
modeling approach for the automatic detection of duplicate bug
reports. We utilize and adapt a probabilistic, generative topic model
called {\em Relational Topic Model (RTM)}~\cite{RTM} for the analysis
and inference of the hidden technical topics within bug reports and
the relation of duplicate reports based on their topics. To support
software evolution as new reports are constantly filed and new
duplication information is available, we extend RTM into {\em
incremental} RTM (iRTM) in which the trained model can be quickly
updated without fully re-training.

%Next, let us detail our model.

