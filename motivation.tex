\begin{figure}[t]
\sf
\small
\textbf{ID}:000002; \textbf{CreationDate}:Wed Oct 10 20:34:00 CDT 2001; \textbf{Reporter}:Andre Weinand

\textbf{Summary}: Opening repository resources doesn't honor type.

\textbf{Description}:Opening repository resource always open the default text editor and doesn't honor any mapping between resource types and editors. As a result it is not possible to view the contents of an image (*.gif file) in a sensible way.
\rm
\caption{Bug report BR2 in Eclipse project}
\label{fig:br1}
\end{figure}



\section{Motivating Example}
\label{sec:example}

Let us begin with an example of duplicate bug reports that
motivate our approach. Generally, a bug report is a record in a
bug-tracking database, containing several descriptive fields about the
bug(s). Important fields in a bug report include 1) a unique
identification number of the report (\textbf{\sf ID}), creation
time (\textbf{\sf CreationDate}), the reporting person (\textbf{\sf
Reporter}), and most importantly, a short summary (\textbf{\sf
Summary}) and a full description (\textbf{\sf Description}) of the
bug(s).

\vspace{0.05in}\noindent\textbf{Observations on a bug report}
Figure~\ref{fig:br1} displays an example of an already-fixed bug
report in Eclipse project. As shown, this bug report was assigned the
ID of 2 and reported on 10/10/2001 by Andre Weinand for a bug on
Eclipse v2.0. It described that the system always used the default
text editor to open and display any resource file (e.g. a GIF image)
stored in the repository despite its type. Analyzing the textual
description, we have the following observations:

%functions/technical aspects/concerns/featues 

\begin{enumerate}

\item This bug report is about two technical functions in Eclipse:
\emph{manipulating} (MAN) and \emph{versioning} (VCM) of software
artifacts. In general, MAN involves the operations such as \emph{open},
\emph{view}, \emph{edit}, and \emph{save} on files/resources. VCM
involves the operations such as \emph{connect}, \emph{commit},
\emph{update}, etc.

\item The bug occurred in the code implementing MAN. That is, the operation
\emph{open} on a resource file in the repository was incorrectly
implemented.
%Technically, the system maintains no mappings between resource types
%and editors, thus, it uses the default text editor to open all kinds
%of resource.

\item In the bug report BR2, the technical function MAN can be recognized
in its contents via the words that are relevant to MAN such as
\code{editor}, \code{open}, \code{view}, \code{content},
\code{resource}, \code{file}, \code{text}, and \code{image}.
Similarly, the description also contains relevant terms to VCM such as
\code{repository}, \code{resource}, and \code{file}. Note that, some
words such as \code{resource} and \code{file} are used to describe
both functions MAN and VCM. If considering bug reports as textual
documents, we can view the described technical functions as the
\textbf{topics} of those documents.

\end{enumerate}

\begin{figure}
\sf
\small
\textbf{ID}:009779; \textbf{CreationDate}:Wed Feb 13 15:14:00 CST 2002; \textbf{Reporter}:Jeff Brown

\textbf{Resolution}:DUPLICATE

\textbf{Summary}: Opening a remote revision of a file should not always use the default text editor.

\textbf{Description}: \code{OpenRemoteFileAction} hardwires the editor
that is used to open remote file to
\code{org.eclipse.ui.DefaultTextEditor} instead of trying to find an
appropriate one given the file's type.

You get the default text editor regardless of whether there are
registered editors for files of that type -- even if it's binary. I
think it would make browsing the repository or resource history
somewhat nicer if the same mechanism was used here as when files are
opened from the navigator. We can ask the Workbench's
\code{IEditorRegistry} for the default editor given the
filename. Use text only as a last resort (or perhaps because of a
user preference).  \rm
\caption{Bug report BR9779, a duplication of bug report BR2 in Eclipse}
\label{fig:br2}
\end{figure}

\vspace{0.04in}\noindent\textbf{Observations on a duplicate bug
report} Figure~\ref{fig:br2} presents bug report ID 9779, filed on
02/13/2002 by a different reporter, Jeff Brown. This report was
determined by Eclipse's developers as reporting the same bug as in
BR2. Analyzing the contents of BR9779 and comparing to those of BR2,
we can see that

%have the following observations:

\begin{enumerate}


\item BR9779 also describes two aspects/functions: MAN - manipulating and
VCM - versioning of software artifacts. MAN was also reported to be
buggy.

\item The terms that are used to describe MAN are similar to those in BR2,
e.g. \code{open}, \code{file}, and \code{editor}. However, the terms
describing VCM are somewhat different, such as \code{remote},
\code{revision}, or \code{history}.

\item BR9779 provides additional information about the bug. It notifies
that \code{OpenRemoteFileAction}, the class responsible for opening
a remote file, is directly associated with
\code{org.eclipse.ui.DefaultTextEditor}, i.e., it always uses the
default editor to open a remote file. The report also provides a
fixing suggestion: asking Workbench's \code{IEditorRegistry} for the
default editor given the filename.
%Based on those two examples and five observations, we imply/conclude
%that:
%1. Duplicate bug reports do exist in real-world software development (the bug discussed here is also duplicately reported in BR000094 and BR015392). This is because the software system tend to be used by many independent people in many different environment and usage settings. Thus, an existing bug is easily seen/experienced and reported by several people.

\end{enumerate}

\vspace{0.03in}\noindent\textbf{Implications} The detection of such
duplicate bug reports has several benefits in software development and
maintenance. First, the duplicate bug reports, reported by people with
different points of view and experience could provide different kinds
of information about the bug(s), thus, help in the debugging and
fixing process. Importantly, detecting duplicate bug reports would
help developers to avoid redundant bug fixing efforts.

However, manual detection of duplicate bug reports is highly
time-consuming. For a large-scale project, the number of bugs and the
bug reporting rate are fairly high. For example, in Eclipse, there are
currently more than 363K bug reports and there are from 2-5 newly
filed bug reports every hour. To detect whether a new bug report is a
duplication of some existing bug report, one would need to analyze and
compare it with all those bug reports, both new and existing
ones. Detection on the whole dataset would result in the analysis of
$O(N)$ and the comparison of $O(N^2)$, with $N$ is the total number of
bug reports. Therefore, an automatic detection of duplicate bug
reports is highly desirable.

The above example shows us that the detection of duplicate bug reports
could be based on their technical topics, rather than the concrete
terms/words that are used. Intuitively, topics are \emph{latent}, {\em
semantic} features, while terms are \emph{visible, textual} features
of the documents. One could expect that the former would describe the
similarity of the documents more accurate than the latter. For
example, BR2 and BR9779 describe the same topic, but they might use
\emph{different} terms for the same topic. In BR9779, the words
\code{remote}, \code{revision}, and \code{history} are used to
describe VCM, while they do not appear in BR2.

%Thus, term-based assessment of those two documents might be less
%accurate/effective than topic-based assessment.

%Actually, using tfidf, a term-weighting scheme, we have calculate the
%cosine similarity of BR000002 and BR009779 to be 0.5???, too small to
%be considered as similar documents. In our approach, the topic-based
%probability that they are duplicated is estimated as 0.8???.
% topic EDIT in BR2 >> topic EDIT in BR1. (Buggy one!!!)

Based on aforementioned observations, we propose to use a topic
modeling approach for the automatic detection of duplicate bug
reports. We utilize and adapt a probabilistic, generative topic model
called {\em Relational Topic Model (RTM)}~\cite{RTM} for the analysis
and inference of the hidden technical topics within bug reports and
the relation of duplicate reports based on their topics. To support
software evolution as new reports are constantly filed and new
duplication information is available, we extend RTM into {\em
incremental} RTM (iRTM) in which the trained model can be quickly
updated without fully re-training.

%Next, let us detail our model.

