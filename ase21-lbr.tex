%\documentclass[conference]{IEEEtran}
\documentclass[10pt,conference]{IEEEtran} 
\IEEEoverridecommandlockouts
% The preceding line is only needed to identify funding in the first footnote. If that is unneeded, please comment it out.
%\usepackage{cite}
%\usepackage{amsmath,amssymb,amsfonts}
%\usepackage{algorithmic}
%\usepackage{graphicx}
%\usepackage{textcomp}
%\usepackage{xcolor}
%\def\BibTeX{{\rm B\kern-.05em{\sc i\kern-.025em b}\kern-.08em
%    T\kern-.1667em\lower.7ex\hbox{E}\kern-.125emX}}

\usepackage{epsf}
%\usepackage[bookmarks=false]{hyperref}

\usepackage{booktabs}
\usepackage{balance}
\usepackage[utf8]{inputenc}
%\usepackage{amsmath}
%\usepackage{systeme}
%\usepackage{amsfonts}
%\usepackage{amssymb}
\usepackage{graphicx}
\usepackage{listings}
\usepackage{algorithm}
\usepackage[noend]{algpseudocode}
\usepackage[utf8]{inputenc}
\usepackage[english]{babel}
\usepackage{xspace}
\usepackage{tabularx}
\usepackage{multirow}
\usepackage[table,xcdraw]{xcolor}
\usepackage{listings}
\usepackage{paralist}
\usepackage{subcaption}
%\usepackage{tikz}
\usepackage[skins]{tcolorbox}
\usepackage{enumitem,kantlipsum}
\newtcolorbox{myframe}[2][]{%
  enhanced,colback=white,colframe=black,coltitle=black,
  sharp corners,
  toprule=1.0pt,
  rightrule=0.3pt,
  leftrule=0pt,
  bottomrule=0pt,
  fonttitle=\itshape\scshape\large,
  left=0pt,right=5pt,top=5pt,bottom=3pt,
  attach boxed title to top right={yshift=-0.3\baselineskip-0.4pt,xshift=-5mm},
  boxed title style={tile,size=minimal,left=0.2mm,right=0.5mm,
    colback=white,before upper=\strut},
  title=#2,#1
}

\newcommand{\TotalPRs}{50}
\newcommand{\Responses}{42}
\newcommand{\NoAnswer}{8}

\newcommand{\Agree}{31}
\newcommand{\AgreeButNotFix}{13}

\newcommand{\Merged}{5}
\newcommand{\Approved}{13}

\newcommand{\CannotFix}{5}
\newcommand{\NotFix}{8}

\newcommand{\Disagree}{11}

\newcommand{\NoChange}{3}
\newcommand{\Override}{2}
\newcommand{\NotFixConvention}{6}

\usepackage{xspace}
\newcommand{\cf}{\hbox{\emph{cf.}}\xspace}
\newcommand{\deletia}{\ldots [deletia] \ldots}
\newcommand{\etal}{\hbox{\emph{et al.}}\xspace}
\newcommand{\eg}{\hbox{\emph{e.g.,}}\xspace}
\newcommand{\ie}{\hbox{\emph{i.e.,}}\xspace}
\newcommand{\st}{\hbox{\emph{s.t.}}\xspace}
\newcommand{\wrt}{\hbox{\emph{w.r.t.}}\xspace}
\newcommand{\viz}{\hbox{\emph{viz.}}\xspace}

\newcommand{\graphsyn}{\textsc{GraSyn}\xspace}
\newcommand{\tool}{\textsc{Phrase2Set}\xspace}
\newcolumntype{L}[1]{>{\raggedright\arraybackslash}p{#1}}
\newtheorem{observation}{Observation}
\newtheorem{property}{Property}
\newcommand{\code}[1]{{\footnotesize\textsf{#1}}}
\usepackage{amsthm}
 \definecolor{dkgreen}{rgb}{0,0.6,0}
\definecolor{gray}{rgb}{0.5,0.5,0.5}
\definecolor{mauve}{rgb}{0.58,0,0.82}

%\newcommand{\code}[1]{{\small\textsf{#1}}}
\newcommand{\algo} {Treed}
\newtheorem{Definition}{Definition}
\newtheorem{Claim}{Claim}
\newtheorem{Lemma}{Lemma}
\newtheorem{Theorem}{Theorem}
\newtheorem{Property}{Property}

\newcommand{\model} {iRTM}

\lstset{
    language={Java}, emph={},
    mathescape=false, escapeinside={/*@}{@*/},
    basicstyle=\scriptsize\sffamily,
    numberstyle=\scriptsize\sffamily,
    emphstyle=\bfseries,
    numbers=left, stepnumber=1, numbersep=-6pt,
    frame=single, xleftmargin=4pt, xrightmargin=4pt, framexleftmargin=0pt, framexrightmargin=0pt,  %xleftmargin=11pt
    columns=flexible, breaklines=true, showspaces=false, showstringspaces=true, showtabs=false, tabsize=2
}

\begin{document}

\title{Relational Topic Model for\\ Duplicate Bug Report Detection}

%\title{Incremental Relational Topic Model for\\ Duplicate Bug Report Detection}

%\author{\IEEEauthorblockN{1\textsuperscript{st} Given Name Surname}
%\IEEEauthorblockA{\textit{dept. name of organization (of Aff.)} \\
%\textit{name of organization (of Aff.)}\\
%City, Country \\
%email address or ORCID}
%\and
%\IEEEauthorblockN{2\textsuperscript{nd} Given Name Surname}
%\IEEEauthorblockA{\textit{dept. name of organization (of Aff.)} \\
%\textit{name of organization (of Aff.)}\\
%City, Country \\
%email address or ORCID}
%\and
%\IEEEauthorblockN{3\textsuperscript{rd} Given Name Surname}
%\IEEEauthorblockA{\textit{dept. name of organization (of Aff.)} \\
%\textit{name of organization (of Aff.)}\\
%City, Country \\
%email address or ORCID}
%\and
%\IEEEauthorblockN{4\textsuperscript{th} Given Name Surname}
%\IEEEauthorblockA{\textit{dept. name of organization (of Aff.)} \\
%\textit{name of organization (of Aff.)}\\
%City, Country \\
%email address or ORCID}
%\and
%\IEEEauthorblockN{5\textsuperscript{th} Given Name Surname}
%\IEEEauthorblockA{\textit{dept. name of organization (of Aff.)} \\
%\textit{name of organization (of Aff.)}\\
%City, Country \\
%email address or ORCID}
%\and
%\IEEEauthorblockN{6\textsuperscript{th} Given Name Surname}
%\IEEEauthorblockA{\textit{dept. name of organization (of Aff.)} \\
%\textit{name of organization (of Aff.)}\\
%City, Country \\
%email address or ORCID}
%}

\maketitle

\begin{abstract}
In software development and maintenance, bug fixing is a
time-consuming, yet unavoidable task. A bug is occasionally reported
by more than one reporters, resulting in duplicate bug reports.
Detecting duplicate bug reports is crucial because it helps reduce the
maintenance efforts from developers as well as provides more
information in the bug fixing process. In this paper, we propose an
automatic approach to this problem. In our approach, a bug report is
considered as a textual document describing one or more technical
aspects of a software system, in which some of them might be
erroneously implemented. The reports similarly describing the same
erroneous technical aspects are considered as duplicate ones. We
utilize Relational Topic Model (RTM), a probabilistic, generative
topic model, to formulate the probabilistic structures of technical
aspects in a collection of bug reports and the duplication indicators
among them. Trained with historical data including identified
duplicate reports, the model can be used to detect other
not-yet-identified duplicate ones.
%To support software evolution, we extend RTM into {\em incremental RTM
%  (iRTM)} in which the trained model can be quickly updated without
%spending a large amount of time for complete re-training when new
%reports are filed or additional duplication information is
%available.
Our empirical evaluation on several large, real-world systems shows
that iRTM outperforms the state-of-the-art approach, achieving up to
90\% top-10 accuracy with up to 8 times faster in updating its
model as new bug reports arrive.
\end{abstract}

%\begin{IEEEkeywords}
%component, formatting, style, styling, insert
%\end{IEEEkeywords}

\section{Introduction}
\label{intro}

%In software development and maintenance, fixing software defects
%(often called bugs) is crucial in software development to produce
%high-quality products. Bug fixing could occur during development or in
%post-release time.

Developers, testers, or users of a software system encounter the
detects and note its incorrect behaviors that do not follow the
requirements or their expectations. They could report such defects in
a bug repository.
%that often called bug-tracking database.
%
%The process of bug fixing continues with the developers analyzing the
%phenomenon, locating the defective code, correcting them, and
%committing the fixed code to the projects' repository.
%
Users could interact with a software system in many different ways,
thus, occasionally, a bug could be filed by multiple reporters. That
leads to the {\em duplicate bug reports} for the same bugs. Because
bug reports keep being filed everyday with high pace (e.g., in
Eclipse, 3-5 new bug reports are often filed every hour), it is~vital
to detect if a new bug report is a duplicate one. This
%Detecting whether a new bug report is a duplicate one is crucial
%because it helps 
helps reduce the maintenance efforts from developers (e.g., if the bug
is already fixed) and provides additional information in the bug
fixing process (e.g., if the bug is not yet fixed).
%
However, such automated detection of duplicate bug reports is not
trivial. With different input data, usage environments or scenarios,
an erroneous behavior might expose as different phenomena (e.g.,
different outputs, traces, or screen views). Moreover, different
reporters might use different terminologies and styles, or report on
different elements to describe the same phenomena. As a result,
duplicate bug reports could be textually dissimilar.

%[Tung: we should show this in the motivating examples]

%To automate the detection process of duplicate bug reports, we propose
%a probabilistic approach. In our approach, each bug report is
%considered as a textual document describing one or more technical
%aspects/functionality of a system, in which some of them might be
%erroneously implemented. The reports similarly describing the same
%erroneous technical aspects are considered as duplicate
%ones. Considering the technical aspects as the topics of those
%text-based bug reports, we utilize Relational Topic Model
%(RTM)~\cite{RTM}, a probabilistic, generative model, to formulate
%their topic structures and duplication indicators. We use Gibbs
%sampling~\cite{gibb} to train the model on historical data with
%identified duplicate bug reports and then detect other
%not-yet-identified duplicate ones.

% iRTM
%With software evolution, new reports are continually filed and new
%duplicate information could also be provided. The trained model should
%be updated with that new information. Therefore, we extend RTM into
%{\em incremental} RTM (iRTM) in which the trained model can be updated
%using a combination of a portion of existing data (historical reports
%and duplication indicators) and newly provided data (i.e., all new
%reports and additional duplication indicators). This helps save time
%on complete re-training on all old and new data, which could be very
%costly as the number of bug reports increases with a relatively high
%pace.

%

%Our empirical evaluation on several large, real-world systems shows
%that {\model} outperforms the state-of-the-art approach from Sun {\em
%et al.}~\cite{davidlo10} in terms of both accuracy and time
%efficiency. It can achieve up to 90\% top-10 accuracy, with updating
%time within 0.14 seconds per new bug report on average. The updating
%time for our model with new data is about 5-8 times smaller than the
%re-training time of the Sun {\em et al.}'s approach
%in~\cite{davidlo10}.

%
%The contributions of this paper include:

%\begin{enumerate}

%\item {\model}, an extended model from RTM to formulate the problem of
%  detecting duplicate bug reports. {\model} captures semantically the
%  technical topics in the bug reports and formulates the semantic
%  similarity measure among duplicate reports based on such topic
%  structures.

%\item Incremental algorithms for {\model} in a) training the model on
%historical bug reports and identified duplications, b) detecting
%not-yet-identified duplicate reports, and c) updating the trained
%model when new data is available. Such incremental solution supports
%well the detection of duplicate bug reports in evolving software.

%\item An empirical evaluation showing the accuracy, scalability, and
%time efficiency of {\model}.

%\end{enumerate}

%The next section presents a motivating example. Section~3 describes
%the details of our model. Section~4 presents the algorithm for
%incremental training of the model and detection of duplicate bug
%reports. Section 5 discusses our evaluation. Related work is in
%Section 6, and conclusions appear last.

%\begin{figure}[t]
%\sf
%\small
%\textbf{ID}:000002; \textbf{CreationDate}:Wed Oct 10 20:34:00 CDT 2001; \textbf{Reporter}:Andre Weinand

%\textbf{Summary}: Opening repository resources doesn't honor type.

%\textbf{Description}:Opening repository resource always open the default text editor and doesn't honor any mapping between resource types and editors. As a result it is not possible to view the contents of an image (*.gif file) in a sensible way.
%\rm
%\caption{Bug report BR2 in Eclipse project}
%\label{fig:br1}
%\end{figure}


\begin{figure}[t]
\sf
\small
\textbf{ID}:488105; \textbf{Reported:}:2016-02-19 08:46 EST; \\\textbf{Reporter}:Marc Dumais

\textbf{Summary}:  [memory] Traditional and Floating Point Renderings: view is initially empty

\textbf{Description}: This happens in both the Memory and Memory Browser views, for Traditional and Floating Point renderings. 

When a new Memory Browser view is created, or when a new rendering is created in the Memory view using one of these renderings, the view is initially not populated. It then becomes populated when a refresh occurs, for example by re-sizing the view. See attached screenshot for an example.
\rm
\caption{Bug report 488105~\cite{bug488105} in Eclipse project}
\label{fig:br1}
\end{figure}


\section{Motivating Example}
\label{sec:example}

Let us explain an example of duplicate bug reports that motivate our
approach. Generally, a bug report is a record in a bug-tracking
repository of a project, containing the descriptions on the
bug(s). Typically, a bug record contains the following important
fields 1) a unique ID of the report (\textbf{\sf ID}), reported date
(\textbf{\sf ReportedDate}), the reporter (\textbf{\sf Reporter}), and
a short summary (\textbf{\sf Summary}) and a full description
(\textbf{\sf Description}).

\vspace{0.05in}\noindent\textbf{Observations on a bug report}
Figure~\ref{fig:br1} displays an example of an already-fixed bug
report in Eclipse project. As shown, this bug report was assigned the
ID of 488105 and reported on 02/19/2016 by Marc Dumais for a bug on
Eclipse. It described that the initial view or initial rendering for
memory browser is not populated. When it becomes populated when a
refresh occurs. Analyzing the textual description, we have the
following observations:

%functions/technical aspects/concerns/featues 

\begin{enumerate}

\item This bug report is about two technical functions in Eclipse:
  \emph{memory browser} (MEM) and \emph{view management} (VM) of
  software artifacts. In general, MEM involves the operations such as
  \emph{create}, \emph{render}, and \emph{populate}. VM involves the
  operations such as {\em refresh}, {\em resize}, {\em rendering}, etc.

\item The bug occurred in the code implementing MAN. That is, the operation
\emph{open} on a resource file in the repository was incorrectly
implemented.
%Technically, the system maintains no mappings between resource types
%and editors, thus, it uses the default text editor to open all kinds
%of resource.

\item In the bug report BR2, the technical function MAN can be recognized
in its contents via the words that are relevant to MAN such as
\code{editor}, \code{open}, \code{view}, \code{content},
\code{resource}, \code{file}, \code{text}, and \code{image}.
Similarly, the description also contains relevant terms to VCM such as
\code{repository}, \code{resource}, and \code{file}. Note that, some
words such as \code{resource} and \code{file} are used to describe
both functions MAN and VCM. If considering bug reports as textual
documents, we can view the described technical functions as the
\textbf{topics} of those documents.

\end{enumerate}

%\begin{figure}
%\sf
%\small
%\textbf{ID}:009779; \textbf{CreationDate}:Wed Feb 13 15:14:00 CST 2002; \textbf{Reporter}:Jeff Brown

%\textbf{Resolution}:DUPLICATE

%\textbf{Summary}: Opening a remote revision of a file should not always use the default text editor.

%\textbf{Description}: \code{OpenRemoteFileAction} hardwires the editor
%that is used to open remote file to
%\code{org.eclipse.ui.DefaultTextEditor} instead of trying to find an
%appropriate one given the file's type.

%You get the default text editor regardless of whether there are
%registered editors for files of that type -- even if it's binary. I
%think it would make browsing the repository or resource history
%somewhat nicer if the same mechanism was used here as when files are
%opened from the navigator. We can ask the Workbench's
%\code{IEditorRegistry} for the default editor given the
%filename. Use text only as a last resort (or perhaps because of a
%user preference).  \rm
%\caption{Bug report BR9779, a duplication of bug report BR2 in Eclipse}
%\label{fig:br2}
%\end{figure}

\begin{figure}
\sf
\small
\textbf{ID}:Bug 493035; \textbf{Reported}: 2016-05-04 18:12 EDT; \textbf{Reporter}:Marc-Andre Laperle

\textbf{Resolution}:DUPLICATE

\textbf{Summary}: [GTK3] CDT Memory Browser initially blank

\textbf{Description}: 
Using I20160504-0035
CDT master as of today
Ubuntu 16.04 and Fedora 23 (GTK 3.18.9)
C/C++ Memory View Enhancements feature installed

1. Create a hello world C++ project
2. Start debugging
3. Open the memory browser
4. Enter main in the address field. The memory view is blank.

If the view is resized, it gets populated correctly. This works correctly in GTK2.
\caption{Bug report 493035~\cite{bug493035}, a duplication of bug report 488105 in Eclipse}
\label{fig:br2}
\end{figure}

\vspace{0.04in}\noindent\textbf{Observations on a duplicate bug
report} Figure~\ref{fig:br2} presents bug report ID 9779, filed on
02/13/2002 by a different reporter, Jeff Brown. This report was
determined by Eclipse's developers as reporting the same bug as in
BR2. Analyzing the contents of BR9779 and comparing to those of BR2,
we can see that

%have the following observations:

\begin{enumerate}


\item BR9779 also describes two aspects/functions: MAN - manipulating and
VCM - versioning of software artifacts. MAN was also reported to be
buggy.

\item The terms that are used to describe MAN are similar to those in BR2,
e.g. \code{open}, \code{file}, and \code{editor}. However, the terms
describing VCM are somewhat different, such as \code{remote},
\code{revision}, or \code{history}.

\item BR9779 provides additional information about the bug. It notifies
that \code{OpenRemoteFileAction}, the class responsible for opening
a remote file, is directly associated with
\code{org.eclipse.ui.DefaultTextEditor}, i.e., it always uses the
default editor to open a remote file. The report also provides a
fixing suggestion: asking Workbench's \code{IEditorRegistry} for the
default editor given the filename.
%Based on those two examples and five observations, we imply/conclude
%that:
%1. Duplicate bug reports do exist in real-world software development (the bug discussed here is also duplicately reported in BR000094 and BR015392). This is because the software system tend to be used by many independent people in many different environment and usage settings. Thus, an existing bug is easily seen/experienced and reported by several people.

\end{enumerate}

\vspace{0.03in}\noindent\textbf{Implications} The detection of such
duplicate bug reports has several benefits in software development and
maintenance. First, the duplicate bug reports, reported by people with
different points of view and experience could provide different kinds
of information about the bug(s), thus, help in the debugging and
fixing process. Importantly, detecting duplicate bug reports would
help developers to avoid redundant bug fixing efforts.

However, manual detection of duplicate bug reports is highly
time-consuming. For a large-scale project, the number of bugs and the
bug reporting rate are fairly high. For example, in Eclipse, there are
currently more than 363K bug reports and there are from 2-5 newly
filed bug reports every hour. To detect whether a new bug report is a
duplication of some existing bug report, one would need to analyze and
compare it with all those bug reports, both new and existing
ones. Detection on the whole dataset would result in the analysis of
$O(N)$ and the comparison of $O(N^2)$, with $N$ is the total number of
bug reports. Therefore, an automatic detection of duplicate bug
reports is highly desirable.

The above example shows us that the detection of duplicate bug reports
could be based on their technical topics, rather than the concrete
terms/words that are used. Intuitively, topics are \emph{latent}, {\em
semantic} features, while terms are \emph{visible, textual} features
of the documents. One could expect that the former would describe the
similarity of the documents more accurate than the latter. For
example, BR2 and BR9779 describe the same topic, but they might use
\emph{different} terms for the same topic. In BR9779, the words
\code{remote}, \code{revision}, and \code{history} are used to
describe VCM, while they do not appear in BR2.

%Thus, term-based assessment of those two documents might be less
%accurate/effective than topic-based assessment.

%Actually, using tfidf, a term-weighting scheme, we have calculate the
%cosine similarity of BR000002 and BR009779 to be 0.5???, too small to
%be considered as similar documents. In our approach, the topic-based
%probability that they are duplicated is estimated as 0.8???.
% topic EDIT in BR2 >> topic EDIT in BR1. (Buggy one!!!)

Based on aforementioned observations, we propose to use a topic
modeling approach for the automatic detection of duplicate bug
reports. We utilize and adapt a probabilistic, generative topic model
called {\em Relational Topic Model (RTM)}~\cite{RTM} for the analysis
and inference of the hidden technical topics within bug reports and
the relation of duplicate reports based on their topics. To support
software evolution as new reports are constantly filed and new
duplication information is available, we extend RTM into {\em
incremental} RTM (iRTM) in which the trained model can be quickly
updated without fully re-training.

%Next, let us detail our model.


\input{formulation.tex}
\section{Algorithms for initial training, detecting, and incremental updating}
\label{algorithm}

This section presents our algorithms for three aforementioned tasks:
initial training, detecting, and incremental updating. Before
presenting the algorithms (Section IV.B), let us describe a core step
in all 3 algorithms, that is, to determine the hidden (latent) topic
assignment $z_d$ of each bug report $d$ based on the provided data,
i.e. the words $w_d$ of each bug report and some recorded duplication
indicators $y_{d,d'}$s. Once the topic assignments ($z_d$s) are
inferred for all documents, we could estimate their topic proportions
($\theta_d$), the per-topic word distribution $\phi_t$ for each topic
$t$, the duplication indicating function $\psi(d,d')$, and thus, the
duplication indicators $y_{d,d'}$ of all pairs of documents.

\subsection{Sampling-based Inference of Hidden Topic Assignment}

%Direct inference of the hidden topic assignment for each document is
%intractable, thus, 

Instead of using variational inference for the hidden topic assignment
as in LDA~\cite{lda} and RTM~\cite{RTM}, we choose an approximate
approach called {\em Gibbs sampling}~\cite{gibb}. That is, the
posterior probability $P(z_d|w_d, y_{d,d'})$ is estimated by randomly
choosing the value for each element $z_{d,n}$ of $z_d$,
element-by-element, based on a distribution calculated from other
sampled values, until $P(z_d|w_d, y_{d,d'})$ is stationary. The choice
of Gibbs sampling also allows us to efficiently perform incremental
training for new data.

%A detailed description of this technique can be found in~\cite{gibb}.

%That is, we estimate the posterior probability/likelihood $P(z_d|w_d,
%y_{d,d'})$ by randomly choosing the value for each
%element $z_{d,n}$ of $z_d$, element-by-element, based on a distribution
%calculated from other sampled values, until such likelihood
%($P(z_d|w_d, y_{d,d'})$) is stationary.

%Once $z_d$ of each document is being sampled, $\theta_d$ and $\phi_t$ are also estimated based on such $z_d$s.

%In this task, the input includes: term vector $w_d$ of each bug
%report, and duplication indicators $y_{d,d'}$ of some pairs of
%manually-identified duplicate bug reports, and chosen parameters
%$\alpha, \beta, \eta$. The output includes: topic assignment vector
%$z_d$ and topic proportion $\theta_d$ of each bug report, per-topic
%term distribution $\psi_t$ of each topic, and duplication indicating
%values $\psi(\theta_d,\theta_d')$ for every pair of reports.

Specifically, the sampling-based inference of $z_d$ for each document
$d$ is as follows. The elements of $z_d$ are initially assigned with
random values. Then, each of its elements $z_{d,n}$ is sampled based
on a conditional distribution calculated from the most recent sampled
values of \emph{all other} elements, denoted by $z_{d,-n}$, and other
given information, i.e. $w_d$ and all $y_{d,d'}$s. Let us use $P(n,t)
= P(z_{d,n}=t|z_{d,-n},w_d,y_{d,d'})$ to denote the probability that
topic $t$ is assigned for the $n^{th}$ position of $d$, given
all such information. This probability depends on:

\begin{enumerate}

\item The assignments in other positions: $P(z_d[n]=t|z_d[-n])$,

\item The word $x$ chosen for position $n$: $P(w_{d,n} = x|z_{d,n} = t,z_{d,-n},w_{d,-n})$,

\item The duplication indicators of $d$ to all other documents:
   $P(z_{d,n} = t|z_{d,-n},z_{d'}, y_{d,d'})$ of all documents~$d'$.

\end{enumerate}

\vspace{0.03in}
\noindent {\bf Computation.} Those probabilities are computed as
follows:

1. $z_{d,n}$, by the generative process, is drawn based on $\theta_d$,
   while $\theta_d$ is drawn from $Dir(\alpha)$ and can be estimated
   based on $z_{d,-n}$. Let us use $\aleph(z,t)$ to denote the count
   function, i.e. the number of elements of vector $z$ having the
   value $t$. Due to the properties of Dirichlet distribution, we have
$$P_1(n,t) = P(z_{d,n} = t|z_{d,-n}) \approx \frac {\aleph(z_{d,-n},t) + \alpha} {\sum\nolimits_{k = 1}^K ({\aleph(z_{d,-n},k)} + \alpha)}$$
$$ = \frac {\aleph(z_{d,-n},t) + \alpha} {N_d - 1 + K.\alpha}$$

In other words, the topic proportion $\theta_d$ of document $d$ is
estimated by counting the assignments on $z_{d,-n}$, smoothened by
$\alpha$. Then, the probability of topic assignment $z_{d,n}$ is computed based
on such proportion.

% the ?expectation? of the distribution of $\theta_d$!!!
%(Tung: I'm not much sure about this).

2. $w_{d,n}$, by the generative process, is drawn based on $\phi_t$ if
   $z_{d,n} = t$. Let us use $w_{d,-n}(t)$ to denote the vector of words
   in $w_{d,-n}$ that are assigned to topic $t$. Due to the properties
   of Dirichlet distribution, we have
$$P_2(n,t)=P(w_{d,n}=x|z_{d,n}=t,z_{d,-n},w_{d,-n})$$ $$\approx \frac
{\aleph(w_{d,-n}(t),x) + \beta} {\sum\nolimits_{y = 1}^V
{(\aleph(w_{d,-n}(t),y) + \beta)}}= \frac {\aleph(w_{d,-n}(t),x) +
\beta} {N^{-}_{d,t} + V.\beta}$$ in which $N^{-}_{d,t}$ is the number
of words in $w_{d,-n}$ that are assigned to topic $t$.
In other words, when the position $n$ is assigned to topic $t$, we
emphasize only to the words in the document at other positions
assigned to topic $t$ (i.e. $w_{d,-n}(t)$), estimate the selection
probability of each word based on $w_{d,-n}(t)$ (smoothened by
$\beta$), and calculate the likelihood that the word $x$ has been
chosen for position $n$ (observed from data) based on that
distribution.

3. $y_{d,d'}$ is drawn based on $\psi(d, d')$ by the generative
   process. Given $z_{d,-n}$ and $z_{d'}$, we can estimate $\theta_d$
   and $\theta_d'$. Thus:
\[
\tiny
P_3(n,t,d') = P(z_{d,n} = t|z_{d,-n},z_{d'}, y_{d,d'}=1) \propto \] \[ \frac {P(y_{d,d'}=1|z_{d,n}=t, z_{d,-n},z_{d'})} {P(y_{d,d'}=1|z_{d,-n},z_d')} = \frac {\psi(d,d')} {\psi(d_{-n},d')}
\]

Since $\theta_d$ and $\theta_{d'}$ could be estimated based on $z_d$ and $z_{d'}$, by definition, $\psi(d,d') \approx exp(S)$ and $\psi(d_{-n},d') \approx exp(S')$ with

$S = {\sum_{k=1}^K(\eta_k.\frac {\aleph(z_d,k)} {N_d}. \frac {\aleph(z_{d'},k)} {N_{d'}} + \nu)}$

\noindent and

$S_{-n} = {\sum_{k=1}^K(\eta_k.\frac {\aleph(z_{d,-n},k)} {N_d}. \frac {\aleph(z_{d'},k)} {N_{d'}} + \nu}))$

\noindent Therefore, 
$$\frac {\psi(d,d')} {\psi(d_{-n},d')} = exp(S - S_{-n})
= exp({\eta_t.\frac 1 {N_d}.\frac {\aleph(z_{d'},t)} {N_{d'}}})$$

\noindent because there is only one difference between $S$ and $S_{-n}$ at position $n$ and $z_{d,n} = t$).
The formula means that, if $d$ and $d'$ are recorded as duplicate
reports, the higher proportion of topic $t$ in $d'$, the higher the
probability that a position in $d$ is assigned to topic $t$.

%[TUNG: We have the last term from reducing the formula of $\psi$.
%Note that: the topic assignments in the [tu so] and [mau so] has only a difference at for topic $t$ at location $n$].

Using all the above, we can calculate the distribution for sampling
topic assignment at each position of $d$ as:

$$P(n,t) \propto P_1(n,t).P_2(n,t). \prod_{d': y_{d,d'}=1} P_3(n,t,d')(*)$$

The last product is applied for all documents initially specified as
the duplications of $d$, i.e. all $d'$s such that $y_{d,d'} = 1$. This
is used for the pairs of reports that were recorded as duplicate
ones. For the documents having no observed duplication indicators to
$d$, we consider their indicators as un-observed, thus, ignore their
impact in topic assignment of $d$.

\subsection{Algorithms for Three Phases}

Let us describe the algorithms for three core tasks in {\model}.

%the usage of
%{\model}.

%^\vspace{0.05in}\noindent\textbf{1. Training with Initial Data}
\subsubsection{Training with Initial Data}

Using the previous core step, the initial training algorithm is as
follows:

\begin{itemize}

\item Step 1. Initialize randomly the values for all $z_d$s.

\item Step 2. For each $d$, sample $z_d$ element-by-element following
distribution (*) until it is stationary.

\item Step 3. Estimate the topic proportion of each document: $\theta_{d,t}
\approx \frac {\aleph(z_d,t)} {N_d}$.

\item Step 4. Estimate the word distribution for each topic: $\phi_{t,x}
\approx \frac {\sum_{d}{\aleph(z_d(t),x)}} {\sum_{d} {N_d(t)}}$,
i.e. count the assignments $N_d(t)$ of topic $t$ in each document $d$,
sum up for the whole collection, and estimate the proportion of
assignments using word $x$.

\end{itemize}

%\vspace{0.05in}\noindent\textbf{2. Detecting Duplicate Bug Reports}

\subsubsection{Detecting Duplicate Bug Reports}

Given a new report $d$, we need to determine if it is
duplicate of a bug report(s) in the historical data. The
detection process is as follows:

\begin{itemize}

\item Step 1. Initialize the values for $z_d$.

\item Step 2. Sample $z_d$ following the distribution (*) until it is
stationary.

\item Step 3. Estimate the topic proportion of $d$: $\theta_{d,t} \approx
\frac {\aleph(z_d,t)} {N_d}$.

\item Step 4. For any other document $d'$ in the collection, calculate the
duplication indicating function $\psi$ on the pair $(d, d')$ to infer
their duplication indicator $y_{d,d'}$.

\end{itemize}

Note that, $d$ could be a bug report in the historical data. In this
case, we could detect the not-yet-identified duplicate reports among
the filed ones, and the steps 1-3 are not needed since they were
done in the training phase.

%\vspace{0.05in}\noindent\textbf{3. Updating with Newly Available Data}

\subsubsection{Updating with Newly Available Data}

%In practice, the bug reports are constantly filed. New information on
%the duplicate reports is also provided. For example, the users could
%manually identify some new duplicate bug reports. They could verify
%some of the automatically detected duplications by our tool as true
%duplications. The model, thus, needs to be updated with this newly
%available information. Otherwise, if the initially trained model is
%used to process new data, that model might not fit well.

%because it is trained with old data, which might be totally irrelevant
%in the new data.

%A naive updating method is to completely re-train the model on both
%already-trained and newly available data. Since the new data is
%provided with high rate and volume, and the trained data is also of
%high volume, this naive approach would be very inefficient. However,
%if we train the model with only new data, we might miss the potential
%duplications between the new and the existing bug reports. Thus, our
%balanced approach is to select a representative portion of existing
%data to re-train with new data and use the trained information to
%update the global parameters (e.g. per-topic word distribution) as
%suggested in~\cite{canini09}.

Based on this strategy, our algorithm for incremental updating is as
follows. The input of our algorithm includes the input and output from
the last training step. In addition, the input also includes new data,
i.e. newly filed bug reports and newly provided duplication
indicators (could be among either recorded or new reports). The output
is similar as in the initial training phase.

\begin{itemize}

\item Step 1. Select all the existing/historical bug reports that are
indicated as duplications via the newly provided duplication indicators --
if exists.

\item Step 2. Randomly select another portion of historical bug reports
until having a total of $r\%$ of existing bug reports. When a bug report
is selected, we also select all its duplicate reports.

\item Step 3. Combine the selected bug reports and newly filed ones, and then
train the model on this data using the algorithm in Section~3.1.

\item Step 4. Re-estimate the parameters of the model.

\end{itemize}

$\theta_d$ is only re-estimated for the reports selected in
re-training. However, $\phi_t$ needs to be re-estimated for all bug
reports. This is done without re-counting the non-selected documents
by storing the counting values from the last training. For example,
let $n_0$, $n_1$, and $n_2$ are the numbers of words $x$ assigned to
topic $t$ in the last trained data, the old data selected for
retraining, and the re-trained data, respectively. The number of words
$x$ assigned to topic $t$ after retraining is $n_0 - n_1 + n_2$. Since
$n_0$ is stored, we only need to count $n_1$ and $n_2$ on the data
used for re-training, which is much smaller than the whole data.

\section{Evaluation}
\label{eval}

In this section, we describe our empirical evaluation on the detection
accuracy of {\model} in comparison with the state-of-the-art,
SVM-based approach by Sun {\em et al.}~\cite{davidlo10}. All of
experiments were carried out on on a computer with CPU AMD Phenom II
X4 965 3.0 GHz, 8GB RAM, and Windows~7.

%We also re-implemented the machine learning approach described in
%their paper~\cite{davidlo10} using SVM in LIBSVM tool.

\subsection{Data Sets and Feature Extraction}

\begin{table}[t]
\centering
\caption{Statistics of All Bug Report Data}
    \begin{tabular}{lcrrr}
    \hline
    Project &  Time period &  Report &  Duplicate &  Term \\
    \hline
    Eclipse  &  06/29/2008 - 06/28/2010 & 6,100 & 981 & 22,558 \\
    OpenOffice & 04/12/2010 - 04/10/2011 & 7,000 & 338 & 22,051 \\
    Firefox  &  01/26/2011 - 04/11/2011 & 20,000 & 936 & 42,515 \\
    Apache  &  11/19/2006 - 03/30/2011 & 10,000 & 494   & 34,850 \\
    FreeDesktop &  01/25/2010 - 04/13/2011 & 10,000 & 543   & 33,068 \\
    NetBeans    & 06/17/2010 - 04/13/2011 & 10,000 & 993 & 27,417\\
    \hline
    \end{tabular}%
\label{data}
\end{table}

We conducted an empirical evaluation of {\model} on several
open-source systems. We collected the data from the bug repositories
of the systems (Table~\ref{data}). Column \code{Time period} displays
the time period of collected bug reports. Columns \code{Report} and
\code{Duplicate} show the numbers of bug reports and duplicate ones,
respectively.
%For Eclipse, we chose Eclipse' s platform component from October 2000
%to July 2010 with 61,110 bug reports, in which 14,020 are determined
%by Eclipse's developers as duplicate ones. For Jazz project from June
%2005 to June 2008, the total number of bug reports are 34,228, in
%which 874 of them are recorded as duplications. 
Each bug report has its unique ID, a summary, a description, comments,
and other metadata (e.g. severity, priority, its reporter, creation
date, platform, etc).


In our experiment, we extracted and merged the summary and description
of each report, and used the merged contents as the document for the
report. Each document was then preprocessed such as stemming for term
normalization, and removing grammatical words (e.g. ``a'', ``the'',
``and'', etc) and those terms appearing once in the entire corpus as
in~\cite{RTM}.
%%%This phase include stemming for term normalization, removing
%%%grammatical words (e.g. ``a'', ``the'', ``and'', etc) and those that
%%%appear once in the entire corpus or appear in almost all
%%%documents.
Then, all the words were collected and indexed into a vocabulary.
Column \code{Term} shows the number of extracted terms in each
vocabulary set after pre-processing. After this phase, a bug report is
represented as a vector of indexes of its words in the vocabulary and
is used in the model. Duplication information among bug reports was
also extracted from the repositories.

%%%That is, a document of bug report $d$ with $N$ words will have the
%%%form ${\bf{w}}_d=(w_{d0}, w_{d1}, ..., w_{dN})$ where $w_{dk}$ is the
%%%index of the word at position $k$ in the vocabulary.
%%%The link indicator for $d$ with another bug report $d'$ will take the
%%%value of 1 if they are duplicate, otherwise, it will take the value of
%%%0. The vectors of bug reports and the values for the link indicators
%%%were used as features in training {\model}.

%This vector and the link indicator of the duplicate reports of $d$
%with all other known bug reports $d'$, which take value of $1$ if $d$
%is a duplicate of $d'$ and $0$ otherwise, will be applied to the input
%of iRTM.

\subsection{Evaluation Setup and Metrics}

For the purpose of comparing the performance of {\model} with that of
other work, we use the same evaluation metrics and setup as in the
previous research of Sun {\em et al.}~\cite{davidlo10} for duplicate
bug report detection. Due to un-availability of their tool, we
re-implemented the SVM-based machine learning approach described in
their paper~\cite{davidlo10} using LIBSVM tool~\cite{libsvm}. The
authors of that work used LIBSVM to implement their approach. They
also used that setup to compare their approach with other
state-of-the-art approaches. The same longitudinal setup was used in
our experiment, simulating the usage of our {\model} tool in
reality. That is, bug reports were sorted according to the
chronological order, and then divided into 10 non-overlapped and
equally sized frames. Each frame was indexed corresponding to their
creation time.

%Thus, bug reports in frame $i-1$ were created before bug reports in
%frame $i$.

Initially, frame $0$ with its bug reports and their duplication
indicators were used for training {\model} (Phase 1). Then, we used
the model to test each bug report $d$ in frame $1$ (Phase 2). Our
model gave a top list of $T$ bug reports that were filed before $d$
(in both frames $0$ and $1$) and were likely to be the duplications of
$d$. If the list contains at least one bug report that is a true
duplication of $d$, we count this as {\em a hit} (i.e. a correct
detection). After that, we updated the model (Phase 3) with new data
in frame $1$, including all bug reports in frame $1$ and all true
duplication indicators within both frames $0$ and $1$. The updated
model was then used to test frame $2$. We continued in the same way
for the remaining frames. For Sun {\em et al.}'s, after each frame, we
completely re-trained the SVM model.
%After each frame, we incrementally updated
%{\model} and re-trained the SVM model in Sun {\em et al.}'s.
At last, we calculated \emph{accuracy}, i.e. the ratio of the number
of hits over the total number of true duplicate bug reports under
test, as in Sun {\em et al.}~\cite{davidlo10}. The {\em top-ranked}
({\em top-T}) accuracy was calculated for each value of cut point $T$
from 1 to 10.



We chose this detection accuracy as a performance metric, rather than
precision (i.e. ratio between the number of correctly detected ones
over the total number of detected ones) and recall (i.e. the ratio
between the number of correctly detected ones over the total number of
duplicate ones). The reason is that using detection accuracy fits
better in evaluating this detection tool: given a new bug report, a
tool returns a list of ranked bug report candidates that could
potentially be a duplication of the given report.
%That is, the goal aimed to evaluate how likely the real duplicate bug
%report of the given bug report is in the top-$T$ results. 
Recall does not reflect well the quality of this type of tool because
the tool can return a very long list of results with the top-ranked
results containing the correct duplicate one of the given bug
report. In such cases, recall is very low, however, the tool is very
useful.


%In this case, data in frame $1$ is updated into the model trained from frame
%$0$. for all frames for each size of the top list from $1$-$20$. Then the trained model is used to test each bug report in frame~$1$. We count the total number of hits
%for correctly detected duplicate reports in frame $1$.
%
%We measure the {\em } of {\model} as follows. If the list contains at least one bug report that is a true duplication of $d$, we consider it
%as {\em a hit}. Accuracy is measured as We consider a
%true duplicate link if it connects two bug reports within the
%corresponding training and testing range. We do not count a link that
%connects a bug report under test with a later bug report because that
%would violate the chronological property.

%After all bug reports in frame $1$ are tested, we use both frames $0$
%and $1$ and the real links of bug reports within those frames for
%training, and continue testing for frame $2$. In this case, data in
%frame $1$ is updated into the model trained from frame $0$. In
%general, after all bug reports in frame $n$ are tested, they are used
%to update the trained {\model} which contains information for the
%frames from $0$ to $n-1$. The new trained model contains the
%information of the frames from $0$ to $n$, and is used to test for
%reports in frame $n+1$. Finally, the overall accuracy for each size of
%top list from $1$-$20$ is computed for all frames.

%---------------------
%The size of a frame was selected as follows. In Eclipse, it was
%reported that there are about 2-5 newly filed bug reports per
%hour. Thus, we choose the frame size of 120, corresponding to the
%number of bug reports per day.  In this experiment, we choose to
%incrementally update the model after one frame and to perform full
%re-training after 7 frames. 



%All experiments were on a computer with Windows 7, Intel Core 2 Duo
%2.5Ghz, 4GB RAM.

%%%In general, the performance is measured by {\em correctness} and {\em
%%%coverage} in duplicate bug reports detection. The correctness is
%%%measured as follows. For each new bug report $d$, {\model} gives a
%%%list of $1$ to $20$ highly possible duplicate bug reports of $d$.  If
%%%the list with size $n$ contains at least one bug report which is a
%%%true duplication of $d$, we consider it as a hit for
%%%{\model}. Correctness is measured as the ratio of the number of hits
%%%over the total number of detected cases. In contrast, coverage is
%%%measured as the ratio of correctly detected duplicate bug reports over
%%%the total number of duplicate ones.

%However, our system can flexibly change for each system both in frame
%size and retraining period. Even we can totally retrain the RTM at
%each time frames because we see that the retraining time for RTM with
%size of 60,000 bug reports is about 3.6 hours using a normal Core 2
%duo laptop.

%\begin{figure}
%\centerline{\epsfxsize=3.6in \epsffile{TopList1.eps}}
%\caption{Recall Rate with Different Top List Sizes}
%\label{recall}
%\end{figure}

%\begin{figure}[h]
%	\includegraphics[width=3.6in,angle=270]{TopList1.eps}
%	\caption{Recall Rate of Eclipse}
%	\label{fig:Toplist1}
%\end{figure}

%Figure~\ref{acc} shows the accuracy result. As seen, for a new bug
%report in Eclipse and Jazz, in 41\% and {\bf 75\%} of the cases
%respectively, {\model} can detect its duplication(s) within a list of
%5 bug reports. With the top list of size 10, {\model} can detect
%correctly 61\% and {\bf 78\%} of the cases. With top list of 20
%reports, the accuracy is up to 72\% and {\bf 79\%} for Eclipse and
%Jazz, respectively. Comparing with Sun {\em et al.}'s
%approach~\cite{davidlo10}, their average accuracy levels at the top
%lists of sizes 5,10, and 20 are 40\%, 58\%, and 63\%.

%------------------------------------------------------------------
%The parameters of {\model} in our experiments were selected after
%fine-tuning for best results: the number of iterations in Gibbs
%sampling is 500 and the number of topics ($K$) is 500.
%------------------------------------------------------------------

%there are many documents classi?ed into the same topic group even
%though they contain other aspects.

\subsection{Sensitivity and Tradeoff Analysis}


In our first experiment, we evaluated the sensitivity of accuracy with
respect to different values for the number of topics ($K$). 
%The number of iterations in Gibbs sampling is 500. 
We ran {\model} on Eclipse dataset for various $K$ values from 10 to
700 topics and measured top-1, top-5, and top-10 accuracy for each
case. Figure~\ref{sensitive} shows the result. As seen, when $K$ is
small, accuracy is very low. This is reasonable because the number of
features for bug reports is too small to distinguish their technical
aspects. When the number of topics increases, accuracy increases as
well and peaks at around 400-500 topics. This peak range depends on
each subject project. However, when $K$ becomes larger, accuracy
starts decreasing because the nuanced topics may appear and topics may
begin to overlap semantically with each other. It causes one
document having many topics with similar proportions. This overfitting
problem degrades the accuracy.

\begin{figure}[t]
\centerline{\epsfxsize=3.3in \epsffile{sensitive.eps}}
\caption{Top-ranked Accuracy with Different Numbers of Topics for Eclipse}
\label{sensitive}
\end{figure}

%%\begin{figure}
%%\centerline{\epsfxsize=3.3in \epsffile{tradeoff.eps}}
%%\caption{Top-ranked Accuracy with Different Numbers of Sampling Size}
%%\label{tradeoff}
%%\end{figure}

\begin{table}[t]
\centering
\caption{Top-ranked Accuracy with Different Sampling Sizes $r$}
\setlength{\tabcolsep}{2.5pt}
\begin{tabular}{|l||r|r|r|r|r|r|r|r|r|r|r|r|r|}
\hline
   r(\%) & 0 & 1 & 5 & 10 & 20 & 30 & 40 & 50 & 60 & 70 & 80 & 90 & 100\\
\hline
   Top-1 (\%) & 47 & 48 & 48 & 49 & 50 & 51 & 51 & 51 & 51 & 51 & 51 & 52 & 52\\
   Top-5 (\%) & 67 & 67 & 69 & 69 & 70 & 70 & 71 & 71 & 71 & 71 & 72 & 72 & 72\\
   Top-10 (\%) & 75 & 75 & 77 & 79 & 79 & 80 & 80 & 81 & 80 & 81 & 81 & 81 & 81\\
\hline
   Time (s) & 17 & 22 & 38 & 58 & 111 & 146 & 190 & 235 & 280 & 330 & 370 & 406 & 477\\
\hline
\end{tabular}
\label{tradeoff}
\end{table}

In our next experiment, we evaluated the sensitivity of accuracy as
the size $r$ of (Gibbs) sampling set varies for incremental updating of
{\model}'s internal data (Section IV.B.3.). We fixed the number of
topics $K$ at 500 because we used Eclipse data set for this
experiment. We varied the size of the sample set from 0, 1, 5, 10,
20, ..., 90, and 100\% of the full size of existing data. The case of
100\% means complete re-training. We measured accuracy and the
updating time for each case. As seen in Table~\ref{tradeoff}, when the
sample size of bug reports used for updating the internal model is
small, time efficiency is gained very much with very little accuracy
reduced. For example, with the selection of 10\% of previous bug
reports for model updating, the updating time is 8.2 times smaller but
top-ranked accuracy reduces only from 2-3\%. This result shows that
our selection strategy for a smaller sample set (Section IV.B.3.) for
incremental updating is efficient and quite accurate. If a small
sample of previous bug reports is selected such that all duplicate
ones are included, accuracy will not reduce much.


\subsection{Accuracy Comparison}

\begin{figure}[t]
\centerline{\epsfxsize=3.3in \epsffile{eclipse3.eps}}
\caption{Accuracy Comparison with Different Top List Sizes for Eclipse}
\label{eclipse}
\end{figure}

In the next experiment, we compared {\model}'s performance with that
of Sun {\em et al.}'s~\cite{davidlo10}. The parameters of {\model} in
this experiment were selected after fine-tuning for best results as
described earlier. Figure~\ref{eclipse} displays the accuracy result
of {\model} in comparison with Sun {\em et al.}'s on Eclipse data
set. As shown, for a new bug report, in half of the detection cases,
{\model} can correctly detect the duplication (if any) with just a
single result. With a list of top 5 resulting bug reports, {\model}
can correctly detect the duplication of a given report in 71\% of the
cases. That is, given a bug report, it can correctly detect its
duplication(s) (if any) within its top-5 recommended bug reports in
71\% of the cases. With top lists of 10 reports, it can correctly
detect in 80\% of the cases. In comparison, Sun {\em et al.}'s tool
can achieve the accuracy levels at the top lists of sizes 5 and 10 at
only 53\% and 58\%, respectively. In general, for top lists from 1-10
bug reports, {\model} achieves higher accuracy than Sun {\em et al.}'s
from 12\%-22\%.

\begin{figure}[t]
\centerline{\epsfxsize=3.3in \epsffile{openoffice.eps}}
\caption{Accuracy Comparison with Different Top List Sizes for OpenOffice}
\label{openoffice}
\end{figure}

Figures~\ref{openoffice}, \ref{firefox}, \ref{apache},
\ref{freedesktop}, and \ref{netbeans} display the accuracy results of
{\model} in comparison with Sun {\em et al.}'s approach on OpenOffice,
FireFox, Apache, FreeDesktop, and NetBeans datasets,
respectively. {\model} consistently achieves very high levels of
accuracy (with up to 63\% for top-1, 87\% for top-5, and 90\% for
top-10 accuracy). On average for each subject system, the top-1,
top-5, and top-10 accuracy levels are 56\%, 78\%, and 85\%,
respectively. For the top-1 to top-10 results, {\model} consistently
outperformed Sun {\em et al.}'s with higher accuracy from 11\%-20\% on
OpenOffice, 4\%-9\% on FireFox, 6\%-11\% on Apache, 18\%-25\% on
FreeDesktop, and 6\%-11\% on NetBeans datasets.


%%Figure~\ref{openoffice} displays the accuracy result of {\model} in
%%comparison with Sun {\em et al.}'s on OpenOffice data set. As seen,
%%{\model} consistently has very high level of accuracy (63\%, 87\%, and
%%90\% for top-1, top-5, and top-10 accuracy). For the top-1 to top-10
%%results, {\model} achieves higher accuracy than Sun {\em et al.}'s
%%approach from 11\%-20\%.

\begin{figure}[t]
\centerline{\epsfxsize=3.3in \epsffile{firefox2.eps}}
\caption{Accuracy Comparison with Different Top List Sizes for FireFox}
\label{firefox}
\end{figure}

%%Figure~\ref{firefox} displays the accuracy result of {\model} in
%%comparison with Sun {\em et al.}'s on FireFox dataset. As shown,
%%{\model} achieves a higher level of accuracy than their approach. With
%%top lists of 5 and 10 reports, it can correctly determine in 77\% and
%%86\% of the cases, respectively. In comparison, the corresponding
%%numbers at top-5 and top-10 in Sun {\em et al.}'s approach is 70\% and
%%77\%. For the top-1 to top-10 results, {\model} achieves higher
%%accuracy than Sun {\em et al.}'s from 4\%-9\%.

\begin{figure}[t]
\centerline{\epsfxsize=3.3in \epsffile{apache.eps}}
\caption{Accuracy Comparison with Different Top List Sizes for Apache}
\label{apache}
\end{figure}

%%Figure~\ref{apache} shows {\model}'s accuracy on Apache dataset. As
%%seen, {\model} achieves higher accuracy than Sun {\em et al.}'s
%%from 6\%-11\%. Importantly, it consistently has very high accuracy
%%(78\% and 83\% top-5 and top-10 accuracy).

\begin{figure}[t]
\centerline{\epsfxsize=3.3in \epsffile{freedesktop.eps}}
\caption{Accuracy Comparison with Different Top List Sizes for FreeDesktop}
\label{freedesktop}
\end{figure}

%%Similarly higher level accuracy was achieved on FreeDesktop dataset
%%for {\model} as shown in Figure~\ref{freedesktop} (77\% top-5 and 84\%
%%top-10 accuracy). For this dataset, {\model} outperformed Sun {\em et
%%al.}'s from 18\%-25\% for top-1 to top-10 lists of results. {\model}
%%also achieves a higher level of accuracy than their approach on the
%%NetBeans dataset (Figure~\ref{netbeans}).

\begin{figure}[t]
\centerline{\epsfxsize=3.2in \epsffile{netbeans.eps}}
\caption{Accuracy Comparison with Different Top List Sizes for NetBeans}
\label{netbeans}
\end{figure}

\subsection{Time Efficiency Comparison}

\begin{table*}[t]
\centering
\footnotesize
\caption{Time Efficiency Comparison}
\begin{tabular}{|l||r|r|r|r||r|r|r|r|}
  \hline
     &  & {\model} & & & & Sun's &  & \\
  \hline
  Project & Initial & Average & Update & Prediction & Initial & Average & Re-Training & Prediction \\
          & Training & Update & per Report  & & Training & Re-Training & per Report  & \\
  \hline
  Eclipse & 850s & 150s & 0.25s & 17s & 810s & 860s & 1.4s & 25s\\
  OpenOffice & 485s & 65s & 0.09s & 13.5s & 334s & 350s & 0.5s & 18s \\
  FireFox & 1,280s & 182s & 0.09s & 43s & 1,350s & 1,420s & 0.7s & 73s\\
  Apache  & 711s & 88.5s & 0.08s & 17.4s & 491s & 522s & 0.53s & 36s \\
  FreeDesktop & 833s & 93.5s & 0.09s & 21s & 546s & 576s & 0.58s & 43s\\
  NetBeans & 953s & 181.5s & 0.18s & 27.5s & 972s & 1,024s & 1s & 67s\\
  \hline
\end{tabular}
\label{timetab}
\end{table*}

%\begin{table}[t]
%\centering
%\footnotesize
%\caption{Time Efficiency Comparison}
%\begin{tabular}{|l||r|r|r||r|r|r|}
%  \hline
%     &  & {\model} & & & Sun's &  \\
%  \hline
%  Project & Initial & Update & Pred. & Initial & Re-Train & Pred. \\
%          & Train &  &  & Train &  & \\
%  \hline
%  Eclipse & 850s & 150s & 17s & 810s & 860s & 25s\\
%  FireFox & 1,280s & 182s  & 43s & 1,350s & 1,420s & 73s\\
%  OpenOffice & 485s & 65s & 13.5s & 334s & 350s & 18s \\
%  \hline
%\end{tabular}
%\label{timetab}
%\end{table}

During running two tools on the collected data sets, we also recorded
the execution time. Table~\ref{timetab} displays the result. Column
\code{Initial Training} shows the amount of initial training time of
the corresponding tool for the first data frame. Column \code{Average
Update} displays the average updating time for {\model} for each data
frame. In contrast, Sun {\em et al.}'s needs to re-train the data and
its average re-training time for each frame is in the column
\code{Average Re-Training}. Column \code{Update per Report} shows the
updating time for each bug report for {\model}, while column
\code{Re-training per Report} is for average re-training time per bug
report in Sun {\em et al.}'s. Column \code{Prediction}
shows the corresponding prediction time for each bug report.

{\model} is much more efficient than Sun {\em et al.}'s SVM approach,
especially for large datasets. {\model} took much shorter time (from
5-8 times faster) for data updating than complete re-training time in
their approach. On average, for a new bug report, it took only about
0.14s for {\model} to update its data.
%The training time in {\model} is proportional to the number of bug
%reports as we will show it later.
The complete re-training time in Sun {\em et al.}'s is in fact higher
than that of its initial training because in later data frames, more bug
reports were included. That is, the more bug reports come, the higher
its complete re-training time will be.

%For the model updating with 120 new bug reports, it took only 1 hour
%in training for {\model}. For a new bug report, it took only 30
%seconds for duplication detection.

\subsection{Scalability}

In our third experiment, we evaluated the scalability of {\model}
for a large data set. In this experiment, we prepared a larger dataset
of Eclipse's bug reports with a longer history. We chose Eclipse' s
platform component from October 2000 to July 2010 with 61,110 bug
reports, in which 14,020 are determined by Eclipse's developers as
duplicate ones.  The sizes of vocabulary sets is 120,372 terms. (In
the previous experiment for comparing {\model} with Sun {\em et al.}'s
approach, we did not use this data set because SVM cannot scale.)

Figure~\ref{time} shows the dependence between training time and the
number of bug reports in Eclipse's dataset. As shown, the amount of
training time increases almost linearly with respect to the number of
bug reports and their duplicate indicators (We keep the number of
iterations of Gibbs sampling the same). This is a key advantage of
{\model} over the SVM-based model in Sun {\em et al.}~\cite{davidlo10}
where the complete training time increases significantly as the number
of bug reports increases. Importantly, the update time with
incremental training is much smaller than that of the complete
re-training, and {\model} still scales well as the number of bug
reports increases.

\begin{figure}[t]
\centerline{\epsfxsize=3.5in \epsffile{time.eps}}
\caption{Training Time for Eclipse's Data}
\label{time}
\end{figure}

%Importantly, to compare the result from an incremental run with that
%of a fully re-training run, we compared the detection results after 7,
%14, 21, and so on frames between two runs. In all cases, both runs
%gave no significantly different results.

%\subsection{Threats to Validity}

\vspace{0.04in}
\noindent {\bf Threats to Validity.}  For comparison, we
re-implemented the SVM technique in Sun {\em et al.}~\cite{davidlo10},
rather using their tool (which is not available). However, we followed
exactly their description in their paper~\cite{davidlo10}, and used
the same machine learning tool LIBSVM~\cite{libsvm} as in their work
to re-implement it.


%The training time is a function of number of bug reports, number of
%duplicate links, number of iterations of Gibbs sampling.
%However, through experiment, the value is almost a linear function of
%the number of bug reports and the number of duplicate links.
%When the number of of duplicate links increases, iRTM ensures that the
%training time not increasing explosively.
%The update time when we use the incremental mechanism is even more
%faster. Figure (~\ref{time}) shows the dependence of update time as a
%function of number of samples used for random Gibbs sampling and the
%size of bug reports.



%\begin{figure}[h]
%	\includegraphics[width=3.6in,angle=270]{TrainingTime1.eps}
%	\caption{Training time for Eclipse}
%	\label{fig:TrainingTime1}
%\end{figure}

%Comparing with Sun {\em et al.}'s approach~\cite{davidlo10}, their
%average accuracy levels at the top lists of sizes 5,10, and 20 are
%41\%, 60\%


\section{Related Work}

A related work to {\model} is the Support Vector Machine (SVM)-based
approach from Sun {\em et al.}~\cite{davidlo10}.  In the training
phase of their model, all the pairs of duplicate bug reports are
formed and considered as the positive samples. All other pairs of
non-duplicate bug reports are used as the negative samples. For each
sample (i.e. a pair of reports), a total of 54 features are
extracted. Each feature is represented by the sum of all inverse
values of the frequencies of documents containing terms or bi-grams
(i.e. two consecutive terms) that appear in the summaries and/or
descriptions of both bug reports in the sample. All positive and
negative pairs/samples are used to train and derive the parameters of
a SVM model. In the prediction phase, as a new report arrives, it
would be paired with all existing reports. Then, each pair would be
fed into the trained SVM model and be classified as positive or
negative. Positive pairs imply duplicate reports. Finally, if a new
bug report is predicted by SVM model as duplicate, it is compared
with existing bug reports to determine its master bug report.

There are key advances of {\model} over their SVM model. First, their
model is not suitable for software evolution. It cannot work in an
{\em incremental} manner. For new bug reports, their model requires
complete re-training. As the project evolves, the bug report data gets
increasingly large, the training set continually grows, and the
re-training time will keep increasing significantly. The reason is
that as more reports are added, the numbers of (negative and positive)
pairs/samples will dramatically increase due to the nature of pairing
in the SVM model. With the speed of 3-5 new bug reports per hour
(e.g. in Eclipse)~\cite{davidlo10}, their time cost of for fully
re-training is much higher than {\model}'s updating time.

%not quite practical. In contrast, {\model} can incrementally update
%its parameters in very short time as new data is available.

Another disadvantage of their approach is that, the SVM model predicts
that a new report is a duplication of multiple existing bug reports
but requires a second phase to rank which one is more likely than
others. That is, after predicting the duplication of a new bug report,
their tool performs a second phase to determine the list of potential
master reports. This adds extra computational time. In contrast,
{\model} is able to rank potential master reports based on the
probabilities of generating corresponding pairs of reports. The reason
is that {\model} treats the problem of detecting duplication reports
as a {\em ranking problem}, while their SVM-based approach considers
it as a {\em classification one}. Moreover, in contrast to our
generative model, their model is SVM, a discriminative model. The
quality of their results depends very much on the positive and
negative sets of samples. Because the percentage of duplicate bug
reports is much smaller than that of non-duplicate ones in the
project, the negative set will grow faster and their approach faces
the issue of un-balance between positive and negative samples. With
the generative approach, {\model} does not have to deal with negative
and positive sample sets. Instead, it decides the probability of
generating a pair of duplicate bug reports.

To overcome that, Sun {\em et al.}~\cite{sun-ase11} introduced REP, a
novel IR technique that extends BM25F to consider the long bug reports
and the meta-data such as the reported product, component, and
version. They showed that REP outperformed the state-of-the-art ML
approaches in both accuracy and efficiency. We did not compare with REP
because it is IR-based while we used machine learning.
%XW
%In this work, we combine BM25F with our novel topic model, T-Model, to
%address the cases where duplicate reports have different terms for the
%same technical issue. To our knowledge, DBTM is the first work in
%which topic-based features are used with IR to support the detection
%of duplicate bug reports.
%
Jalbert and Weimer~\cite{weimer08} use a binary classifier model for
predicting duplicate bug reports. They utilizes a linear regression
over {\em textual} features of bug reports computed from the
frequencies of terms in bug reports.
%%%To make a binary classifier, they specify an output value cutoff over
%%%such features that distinguishes between duplicate and non-duplicate
%%%status.
Similar to Sun {\em et al.}'s model, this model requires complete
re-training for new bug reports.
%, which is not quite efficient in practice.
Moreover, their model relies solely on {\em textual similarity}, while
{\model} focuses more on the underlying {\em technical topics} of bug
reports to determine the duplications.
%Finally, their model is discriminative, thus, facing the issue of
%unbalanced positive and negative sample sets of bug reports.

One of the first techniques to detect duplicate bug reports is Runeson
{\em et al.}'s~\cite{runeson07}. In contrast to aforementioned machine
learning (ML) approaches, Runeson {\em et al.}  utilize a natural
language processing (NLP) approach.
%The bug reports are parsed, stemmed, and the stopwords are
%removed.
Each report is modeled by a vector of textual features. The
feature of such a vector at a position corresponding to a term is
computed based on Term frequency-Inverse document
frequency~\cite{salton73}. Vector similarity is used to measure the
similarity among bug reports.
%Given a bug report under investigation, their tool returns similar bug
%reports based on the vector similarities between the new report and
%the existing ones.
Hiew~\cite{hiew06}'s approach for duplicate bug report detection is
based on information retrieval (IR) as in Runeson's. However, it uses
incremental clustering for further grouping of duplicate reports.
%is based on incremental clustering, which is quite similar to
%information retrieval. The main difference is that Hiew�s approach
%further considers the detected duplicate bug-report pairs/groups as
%clusters. Thus, when calculating similarities between a new report
%and existing bug reports, each detected cluster is considered as a
%whole rather than as several individual existing bug reports. That
%is to say, for each detected cluster, this approach will calculate one
%similarity between the new report and the detected cluster instead
%of calculating several similarities between the new report and all
%the reports in the cluster.
Comparing to those approaches, {\model} operates at a higher
abstraction level by comparing the underlying technical topics in
reports, instead of their terms. Moreover, ML approaches have been
shown to outperform NLP/IR approaches~\cite{davidlo10}. Wang {\em et
al.}~\cite{taoxie08} combine NLP with execution trace information in a
report.
%%%They utilize both Tf and Idf for textual feature extraction.
DBTM~\cite{ase12} uses a combination of IR and topic modeling.
We do not use IR in this work, therefore, we did not compare our work
with DBTM.
%Despite performance improvement, their approach is not always
%applicable in the cases where execution information considered in
%their tool is not available.
%%%and hard to collect,
%%%especially for binary programs. Sun {\em et al.}~\cite{davidlo10}
%%%reported that the percentage of reports having execution information
%%%is very low (0.83\%).

Other researchers also focus on bug reports. It is suggested that
duplicate bug reports complement to one another to help
in bug fixing~\cite{bettenburg-icsm-2008}. 
%%Bettenburg {\em et al.}~\cite{rahul08} analyzed information mismatch
%%between what developers need and what users supply to determine good
%%properties in bug reports. 
Structural information from bug reports has been shown to be
useful~\cite{bettenburg-msr08,ko06}.
%%Hooimeijer and Weimer~\cite{weimer-ase07} develop a statistic-based
%%model to automatically predict the quality of bug reports.
%Ko {\em et al.}~\cite{ko06} perform linguistic analysis on bug reports
%and suggest more structure for their contents.
Other researchers categorize bug reports based on types, quality, or
severity~\cite{rahul08,weimer-ase07,anvik06,andy-pod03,cubranic04,menzies08,bettenburg-eclipse07,weimer06,lucca02,fischer03,Sandusky04bugreport}.
%However, none of them addresses the automatic detection of duplicate
%bug reports.
From bug reports, prediction tools~\cite{weiss07,skim06} can tell
whether a bug could be resolved with certain fixing time. Approaches
for automatic assignments of bug fixers include
~\cite{anvik06,Canfora05howsoftware}.
%canfora-sac06}.
%%Other approaches aim to study the relationships among bug
%%reports~\cite{fischer03,Sandusky04bugreport}. 
Gethers and Poshyvanyk~\cite{gethers10} utilize RTM in capturing the
latent topics in classes and their relationships.





%and Whitehead claim that the time it takes to fix a
%bug is a useful software quality measure [8]. They measure
%the time taken to fix bugs in two software projects. We
%predict whether a bug will eventually be resolved as a duplicate
%and are not focused on particular resolution times or
%the total lifetime of real bugs.





\section{Conclusion}

We propose a probabilistic model for detecting duplicate bug
reports. 
%%Each bug report is considered as a textual document about
%%technical aspects of a system. 
Duplicate bug reports are the ones similarly describing the same buggy
technical topics. We adapt RTM to formulate the probabilistic
structures of technical aspects in a collection of bug reports and the
duplication indicators among them.
%Trained with prior data on identified duplicate reports, the model is
%used to detect not-yet-identified duplicate ones. 
We also extend RTM into iRTM in which the trained model can be quickly
updated. Our evaluation on real-world systems shows that iRTM is more
accurate and time efficient than the state-of-the-art approach in Sun
{\em et al.}~\cite{davidlo10}.

%to formulate the probabilistic structures of
%technical topics in the collection of bug reports, and to find the
%indicators of the duplication among them based on those topic
%structures. 


\newpage

\balance

%\bibliographystyle{plain}
%\bibliographystyle{ACM-Reference-Format}
\bibliographystyle{IEEEtran}

\bibliography{icsme18}

%\section*{Acknowledgment}

%The preferred spelling of the word ``acknowledgment'' in America is without 
%an ``e'' after the ``g''. Avoid the stilted expression ``one of us (R. B. 
%G.) thanks $\ldots$''. Instead, try ``R. B. G. thanks$\ldots$''. Put sponsor 
%acknowledgments in the unnumbered footnote on the first page.


\end{document}
