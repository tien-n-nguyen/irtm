 %\subsection{Interesting Case Studies}

%Let us discuss a few real-world examples that we found during our
%experimental studies.



%Figure {\ref{bugreport225337}} shows two duplicate bug reports
%detected by {\model}. Except the terms NPE
%(\code{NullPointerException}) and \code{StructuredViewer}, which are
%popular and common in the project, the two reports contain several
%different terms because the reporters found the bug in two different
%usage scenarios. That leads to different exception traces: one
%involving image updating, and another on widget selection. We noticed
%that when running BM25F model by itself, bug report \#225169 is ranked
%8$^{th}$ in the list that could be duplicate of bug report \#225337
%due to the dissimilarity in texts. However, after extracting topics
%via the co-occurrences of other terms such as \code{startup},
%\code{first time}, \code{RSE perspective}, \code{wizard}, etc in the
%previous duplicate reports (e.g. from bug report \#218304, not shown),
%{\model} ranked it at the highest position and detected them as
%duplicate ones.

%\begin{figure}
%\vspace{2mm}
%\sf
%\small
%
%\textbf{Bug Report \#225169}\\
%\textbf{Summary}:  Get NPE when startup RSE on a new workspace\\
%\textbf{Description}:\\
%	Using Eclipse M5 driver and RSE I20080401-0935 build. Start
%	eclipse on a new workspace, and switch to RSE perspective. I
%	could see the following error in the log. But
%	otherwise, things are normal.\\ java.lang.NullPointerException
%	at\\
%	org.eclipse.....getImageDescriptor(SystemView...java:123)\\
%	...\\ at
%	org.eclipse....doUpdateItem(AbstractTreeViewer.java:1010)\\ at
%	org.eclipse....doUpdateItem(SafeTreeViewer.java:79)\\ at
%	org.eclipse....run(StructuredViewer.java:466)...\\
%	-----------------------------------------------------------------------\\
%\textbf{Bug Report \#225337}\\
%\textbf{Summary}:  NPE when selecting linux connection in wizard for the first time\\
%\textbf{Description}:\\
%	After starting an eclipse for the first time, when I went select Linux in the
%	new connection wizard, I hit this exception.  When I tried again a few times
%	later, I wasn't able to hit it.\\
%	java.lang.NullPointerException at\\
%	org.eclipse....getAdditionalWizardPages(RSEDefault...:404)\\
%	...\\
%	at org.eclipse....updateSelection(StructuredViewer.java:2062)\\
%	at org.eclipse....handleSelect(StructuredViewer.java:1138)\\
%	at org.eclipse....widgetSelected(StructuredViewer.java:1168)...\\
%\caption{Duplicate Bug Reports in Eclipse}
%\label{bugreport225337}
%\end{figure}

%-------------------------------------------------------------------------
% SECOND EXAMPLE
%-------------------------------------------------------------------------

\begin{figure}[t]
%\scriptsize
%\sf
\small
\textbf{Bug Report \#240790}\\
\textbf{Summary}: [search] callers are not found when caller and callee reside in sibling Java projects\\
\textbf{Description}:\\
	1. create a java project "common" having one interface\\...
	6. Client.init() is not found being a caller (search cope might be
	"Workspace" or "Hierarchy")\\
-----------------------------------------------------------------------\\
\textbf{Bug Report \#250454}\\
\textbf{Summary}:   [search] Cannot find method references between projects\\
\textbf{Description}:\\
	1) define 3 projects: rootProject, subProject1, and subProject2.\\
	2) set project build depedencies such that subProject1 and subProject2 depend
	on rootProject...\\
	6) Search for references to Square.f() directly from it's declaration in
	Square.java.\\
			6a) EXPECT: one result: ShapeUser.useShape(Shape)\\
			6b) ACTUAL: no results\\
	If ShapeUser is instead located in the direct derivable project dependency
	graph for subProject1, the results are ok....\\
			1) If Square were defined in rootProject itself, the search succeeds.\\
			2) If Square were defined in subProject2, the search succeeds. \\
			3) If subProject2 has a build dependency on subProject1, the search
	succeeds.\\

	Additionally, searching for references to f() from the Shape declaration
	succeeds as defined above.  It appears the search does not go to all dependent
	projects of the project declaring the interface....\\
\caption{Duplicate Bug Reports \#250454 and \#240790}
\label{bugreport250454}
\end{figure}

\vspace{0.04in}
\noindent {\bf Interesting Case Studies} An example of detected duplicate bug reports 
is in Figure~\ref{bugreport250454}. The texts in two summaries are
different. The re-producing steps described by two bug reporters are
totally different. As running with the Sun {\em et al.}'s model alone, bug
report \#240790 was ranked 6th in the list of potential duplications
of bug report \#250454. Trained with historical data via identified
duplicate reports of \#250454 (not shown), DBTM can learn sets of
different terms describing the same technical issue in this new
report, which is the issue of searching objects dependent on sibling
projects.

%Final
%More examples are shown on our project's web site.
%(\url{http://home.engineering.iastate.edu/~anhnt/Research/DBTM/})

%, real duplicate group (bug report \#240790) of bug report \#250454 is
%ranked sixth in the ranked list of BM25f model because the two bugs do
%not has much text similarity (they only share important word
%\code{search}).  The duplicate group is ranked first in our experiment
%using DBTM, since the model can learn that the two bug reports
%\#240790 and \#250454) infer to the issue when searching objects on
%dependent (sibling) projects.

%We also find hundreds of cases where DBTM can elevate the ranked of
%real duplicate group in the ranked list of potential duplicate groups
%of a bug report, in comparison with BM25f model. See our website for
%our finding.
